\documentclass[UTF8]{ctexart}
\usepackage{amsmath, amssymb, amsthm, bm}
\usepackage[ruled,vlined]{algorithm2e}
\SetKwInOut{Input}{输入}
\SetKwInOut{Output}{输出}
\usepackage{float}
\usepackage{graphicx}
\usepackage{tikz}
\usetikzlibrary{arrows.meta,positioning,shapes.geometric}
\usepackage{subcaption}
\usepackage{geometry} \geometry{a4paper, margin=1in}
\usepackage{setspace}
\usepackage{cases}
\usepackage{mathtools}
\usepackage{enumitem}
\usepackage{graphicx}

\newtheorem{definition}{定义}

% Define floor function if needed (already in mathtools?)
% \DeclarePairedDelimiter{\floor}{\lfloor}{\rfloor}

\begin{document}
%% ADDED: Section for Introduction placeholder
\section{引言}

\subsection{研究背景}

随着5G/6G无线通信技术的快速发展和智能车联网(Connected and Autonomous Vehicles, CAV)应用的蓬勃兴起,车辆边缘计算(Vehicular Edge Computing, VEC)已成为解决车联网高计算需求与有限车载计算资源矛盾的关键技术范式\cite{ref1}。在VEC系统中,车辆产生的计算密集型任务(如高精度地图构建、实时目标检测、路径规划等)可以通过无线通信网络卸载至路边基础设施单元(Road Side Units, RSUs)、无人机(Unmanned Aerial Vehicles, UAVs)等边缘计算节点进行处理,从而有效缓解车载计算资源不足的问题,同时满足低延迟、高可靠性的应用需求。

然而,传统的VEC系统架构在面对日益复杂的车联网应用场景时,暴露出诸多限制。首先,车辆高速移动性导致的网络拓扑动态变化使得任务分配和资源调度变得复杂;其次,不同类型计算任务具有异构的服务质量(Quality of Service, QoS)需求,从极度延迟敏感的紧急制动控制(≤2ms)到延迟容忍的大数据分析任务(≥100ms),对系统的智能决策提出了更高要求;再次,边缘计算节点的计算资源有限且负载动态变化,需要高效的任务迁移和负载均衡机制来应对突发计算需求。

\subsection{研究动机与挑战}

针对上述问题,本文提出了一个基于多智能体深度确定性策略梯度的任务迁移与智能缓存(Multi-Agent Deep Deterministic Policy Gradient for Task Migration and Intelligent Caching, MATD3-MIG)系统。该系统旨在通过深度强化学习技术,实现VEC环境中的智能任务分配、动态任务迁移和边缘缓存管理的联合优化。

然而,设计这样一个系统面临以下关键挑战:

\textbf{挑战1:多维度异构任务的智能分类与卸载策略}。车联网应用涵盖了从毫秒级的安全关键任务到秒级的信息娱乐任务,如何根据任务的延迟容忍度、计算复杂度和优先级进行智能分类,并设计相应的卸载导向策略,是系统设计的首要挑战。

\textbf{挑战2:动态环境下的低中断任务迁移}。在车辆高速移动和网络拓扑快速变化的场景下,如何实现边缘计算节点间的无缝任务迁移,同时最小化服务中断时间(downtime)和维持服务连续性,是技术实现的核心难点。

\textbf{挑战3:多优先级队列系统的排队时延预测}。VEC系统中的计算节点需要同时处理多种优先级的任务,如何建立精确的M/M/1非抢占式优先级队列模型来预测排队时延,并据此进行智能决策,是系统优化的关键环节。

\textbf{挑战4:边缘缓存与计算的协同优化}。如何设计智能缓存策略,结合内容流行度预测、协作缓存和背包算法来最大化缓存命中率,同时与任务迁移机制协同工作以提升整体系统性能,是系统集成的重要挑战。

\textbf{挑战5:多目标约束下的联合优化问题求解}。VEC系统需要在任务完成时延、系统总能耗和数据丢失率之间进行权衡,如何将这一复杂的混合整数非线性规划(MINLP)问题转化为可学习的强化学习框架,是算法设计的核心挑战。

\subsection{主要贡献}

为应对上述挑战,本文的主要贡献如下:

\textbf{贡献1:基于延迟容忍度的四级任务分类框架}。提出了一种基于QoS延迟要求的任务分类方法,将计算任务划分为极度延迟敏感型、延迟敏感型、中度延迟容忍型和延迟容忍型四个类别,并为每类任务设计了相应的候选节点集合和卸载导向策略,有效提升了任务分配的精准性。

\textbf{贡献2:低中断的Keep-Before-Break任务迁移机制}。设计了一种新颖的任务迁移框架,通过目标节点预热、增量同步、短静默切换和失败回滚等工程化方法,将传统迁移机制的服务中断时间降低了约60\%,显著提升了服务连续性。

\textbf{贡献3:多优先级生命周期队列模型与时延预测}。建立了基于任务剩余生命周期和优先级的二维队列系统,结合M/M/1非抢占式优先级队列理论,提出了精确的排队时延预测模型,为智能决策提供了理论基础。

\textbf{贡献4:协作式智能边缘缓存策略}。提出了融合历史热度、时间槽热度和Zipf流行度分布的缓存策略,结合邻居协作和背包优化算法,实现了缓存命中率的显著提升(平均提升25\%)和带宽资源的有效节省。


\subsection{文章结构}

本文的其余部分组织如下:第2节建立了VEC系统模型,详细描述了网络拓扑、节点模型和多优先级队列系统;第3节提出了基于延迟容忍度的任务分类框架;第4节介绍了处理模式评估方法;第5节给出了通信与计算模型的数学表达;第6节阐述了任务迁移和边缘缓存的具体实现机制;第7节将系统问题形式化为优化问题并给出了求解框架。图\ref{fig:system_architecture}展示了本文提出的MATD3-MIG系统的整体架构设计。

\begin{figure}[htbp]
    \centering
    \includegraphics[width=0.9\textwidth]{image.png}
    \caption{MATD3-MIG系统架构总览。该图展示了车辆边缘计算环境中的三层异构计算架构:(1)~车辆层:包含多个移动车辆节点,具备基本的计算和通信能力;(2)~路边基础设施层:部署RSU提供边缘计算服务和缓存功能;(3)~空中辅助层:UAV作为灵活的计算和中继节点。在仿真场景中,使用12辆车辆,6个RSU,以及2个无人机.图中还展示了任务卸载、智能迁移、协作缓存的数据流向,以及多智能体强化学习框架中各智能体的决策交互过程。虚线表示潜在的任务迁移路径,实线表示主要的数据传输链路,突出了系统的分布式协作特性。}
    \label{fig:system_architecture}
\end{figure}

\newpage

\section{系统模型}
% 系统模型描述VEC-RL环境中的基本组件和数学建模

\subsection{网络与任务模型}
% 定义VEC系统的基本网络拓扑和任务特征
\begin{itemize}
    \item \textbf{网络节点定义}: 车辆集合 $\mathcal{V} = \{v_1, v_2, \ldots, v_{|\mathcal{V}|}\}$, RSU基站集合 $\mathcal{R} = \{r_1, r_2, \ldots, r_{|\mathcal{R}|}\}$, UAV无人机集合 $\mathcal{U} = \{u_1, u_2, \ldots, u_{|\mathcal{U}|}\}$。
    \item \textbf{计算节点集合}: $\mathcal{N} = \mathcal{V} \cup \mathcal{R} \cup \mathcal{U}$,表示所有可执行计算任务的节点。
    \item \textbf{任务集合}: $\mathcal{J} = \{j_1, j_2, \ldots\}$,表示系统中的所有计算任务。每个任务 $j \in \mathcal{J}$ 具有以下属性:
        \begin{itemize}
            \item $D_j$: 任务输入数据大小 (bits)
            \item $C_j$: 任务处理所需计算量 (CPU cycles)
            \item $c = C_j / D_j$: 计算密度 (cycles/bit)
            \item $S_j$: 任务输出结果大小 (bits)
            \item $T_{max,j}$: 任务最大可容忍延迟 (时隙数)
            \item $\lambda_j'$: 任务类型 $j$ 的平均到达率 (tasks/时隙)
            \item $v_j$: 生成任务 $j$ 的源车辆
        \end{itemize}
        即端到端时延约束为: $T_{total,j} \le T_{max,j} \Delta t$ (秒)
    \item $\Delta t$: 系统时隙持续时间 (秒)
\end{itemize}

\subsection{场景与基站部署说明}
图~\ref{fig:system_architecture} 所示区域为单一主干道路场景,沿线设置两个十字路口。系统中共有\textbf{4 个固定 RSU} 与 \textbf{2 架固定悬停 UAV},且两架 UAV 分别悬停在两个路口正上方。部署与功能如下:
\begin{enumerate}
    \item \textbf{RSU-1(左侧路口路侧)}: 位于左侧十字路口的路侧位置,覆盖路口及相邻路段,提供低时延卸载入口。
    \item \textbf{RSU-2(中段路侧,中央RSU)}: 位于两个路口之间的路段路侧,作为系统\textit{中央RSU} 平衡中段接入流量,并作为回程/迁移链路锚点与全局调度中心(图中红色方框标注)。
    \item \textbf{RSU-3(右侧路口路侧)}: 位于右侧十字路口的路侧位置,服务路口高并发接入与任务卸载。
    \item \textbf{RSU-4(下游端路侧)}: 位于道路下游端的路侧位置,补充覆盖并承接下游车流的卸载需求。
    \item \textbf{UAV-A(左侧路口上空)}: 固定悬停于左侧十字路口正上方,提供空中覆盖与低时延中继,承接时延敏感或拥塞时的任务。
    \item \textbf{UAV-B(右侧路口上空)}: 固定悬停于右侧十字路口正上方,在遮挡场景下提供空中中继/``准移动基站''能力,并保障迁移备份路径。
\end{enumerate}

\textbf{缓存角色划分}: \textit{中央RSU} 负责高复用结果的集中缓存与统一分发;其余 RSU(左右路口与下游端)仅缓存命中概率高的轻量结果;UAV 仅保留极低时延推理所需的少量关键中间/最终结果,以避免其有限能量与存储被占满。

% 描述VEC系统中各类计算和通信节点的基本属性和能力
\begin{itemize}
    \item \textbf{通用节点属性}: 对于任意节点 $n \in \mathcal{N}$,具有以下基本属性:
        \begin{itemize}
            \item $f_n$: 计算能力 (CPU cycles/秒) % 节点的处理器频率
            \item $P_{tx,n}$: 无线传输功率 (Watts) % 节点的发射功率
            \item $B_n^{max}$: 可用总带宽 (Hz) % 节点的通信带宽容量
        \end{itemize}
    \item \textbf{UAV特殊属性}: UAV $u \in \mathcal{U}$ 具有固定悬停位置 $\text{pos}_u = (x_u, y_u, h_u)$ % UAV的三维坐标位置(x,y,高度)
    \item \textbf{RSU 缓存 (RSU $r \in \mathcal{R}$)}:
    \begin{itemize}
        \item 缓存决策变量: $z_{j,r} \in \{0, 1\}$。$z_{j,r} = 1$ 表示 RSU $r$ 缓存了任务 $j$ 的处理结果 $S_j$,$z_{j,r} = 0$ 表示未缓存。
        \item 缓存容量: 每个 RSU $r$ 有一个最大缓存容量 $S_{cache,r}$ (bits)。约束为 $\sum_{j \in \mathcal{J}} z_{j,r} S_j \leq S_{cache,r}$。
            \item 结果请求概率预测: RSU $r$ 可以预测车辆请求任务 $j$ 结果的概率 $P_{req, j, r}(t)$。该概率通过一个逻辑回归模型估算,以确保输出值在 $[0, 1]$ 区间内:
            \begin{equation}
                P_{req,j,r}(t) = \frac{1}{1 + e^{-(\alpha_0 + \alpha_1 H_j + \alpha_2 \lambda_{v_j,req} + \alpha_3 F_{t} + \alpha_4 R_{area})}}
            \end{equation}
            其中:
            \begin{itemize}
                \item $\sigma(x) = 1/(1+e^{-x})$ 是标准的 Logistic (Sigmoid) 函数。
                \item $\alpha_0$ 是模型的偏置项(baseline 请求概率)。
                \item $\alpha_1 \sim \alpha_4$ 是模型学习到的权重系数,衡量每个特征对请求概率对数几率的影响。
                \item $H_j$ 为任务 $j$ 的历史请求频率。
                \item $\lambda_{v_j,req}$ 是生成任务 $j$ 的车辆 $v_j$ 对此类结果的平均请求率。
                \item $F_t$ 是表征时间因素的特征(例如,区分高峰期与低峰期)。
                \item $R_{area}$ 是通过区域功能属性(如POI密度)预测的潜在复用价值。
            \end{itemize}
        \item 缓存命中率: 定义 RSU $r$ 的缓存命中率 $H_{hit,r}$:
        \[
            H_{hit,r} = \frac{\sum_{j \in \mathcal{C}_r} P_{req, j, r}(t) S_j}{\sum_{j \in \mathcal{J}_{req,r}} S_j}
        \]
        其中 $\mathcal{C}_r = \{j \in \mathcal{J} \mid z_{j,r}=1\}$ 是 RSU $r$ 上已缓存结果的任务集合,$\mathcal{J}_{req,r}$ 是 RSU $r$ 服务范围内可能被请求结果的任务集合。该定义为数据量加权的期望缓存命中率。
        (注:标准的缓存命中率通常定义为 命中请求数 / 总请求数。)
        \item \textbf{缓存收益 (时延节省)}: 缓存命中带来的主要收益是时延节省:
        \begin{equation} \label{eq:cache_delay_saving}
            \Delta T_{j,r} = T_j^{\text{no-cache}} - T_j^{\text{cache}}
        \end{equation}
        其中 $T_j^{\text{cache}}$ 通常仅包含结果下载时延 (可能加上一个极短的请求传输时延),显著快于完整的计算流程。
    \end{itemize}
\end{itemize}

\subsection{多优先级生命周期队列模型}
% 基于任务生命周期和优先级的分层队列系统,实现高效的任务调度和资源管理

VEC系统采用多维队列结构,结合任务生命周期和优先级机制,实现任务的高效调度。每个计算节点维护一组分层队列,用于跟踪任务的剩余生命周期和处理优先级。

\begin{itemize}
    \item \textbf{队列维度定义}:
    \begin{itemize}
        \item $l \in \{1, 2, \ldots, L\}$: 任务剩余生命周期索引 % $l=1$表示必须在当前时隙处理
        \item $p \in \{1, 2, \ldots, P\}$: 任务优先级 ($p=1$为最高优先级, $p=P$为最低优先级)
    \end{itemize}

    \item \textbf{车辆队列结构}: 每个车辆 $v \in \mathcal{V}$ 维护完整的队列矩阵
    \begin{equation}
        Q_{v} = \{q_{v,l,p} \mid 1 \le l \le L, 1 \le p \le P\}
    \end{equation}
    其中 $q_{v,l,p}$ 表示车辆 $v$ 中生命周期为 $l$、优先级为 $p$ 的任务数据量 (bits)

    \item \textbf{RSU队列结构}: 每个RSU $r \in \mathcal{R}$ 维护 $(L-1) \times P$ 个队列
    \begin{equation}
        G_r = \{g_{r,l,p} \mid 1 \le l \le L-1, 1 \le p \le P\}
    \end{equation}
    % RSU不需要$l=L$队列,因为任务到达RSU时生命周期已减1

    \item \textbf{UAV队列结构}: 每个UAV $u \in \mathcal{U}$ 维护 $(L-1) \times P$ 个队列
    \begin{equation}
        U_u = \{u_{u,l,p} \mid 1 \le l \le L-1, 1 \le p \le P\}
    \end{equation}
    % UAV队列结构与RSU相同,不包含$l=L$队列
\end{itemize}

\textbf{队列系统基本假设}:
\begin{itemize}
    \item \textbf{到达过程}: 各节点的任务到达过程服从参数为 $\lambda$ 的泊松过程 % 随机到达模型
    \item \textbf{服务过程}: 任务服务时间服从参数为 $\mu$ 的指数分布 % 马尔可夫服务过程
    \item \textbf{队列模型}: RSU和UAV采用 $P$ 个优先级的 \textbf{M/M/1非抢占式优先级队列模型}
    \begin{itemize}
        \item M/M/1: 泊松到达/指数服务/单服务器
        \item 非抢占式: 低优先级任务处理中不被高优先级任务中断
        \item 车辆本地处理队列采用相同模型
    \end{itemize}

    \item \textbf{调度策略}: 系统采用\textbf{非抢占式优先级调度}
    \begin{itemize}
        \item 任务一旦开始处理,不会被中断
        \item 高优先级任务优先获得服务机会
        \item 同优先级任务按FIFO顺序处理
    \end{itemize}

    \item \textbf{容量限制}: 各节点队列容量有限,超出容量的任务将被丢弃

    \item \textbf{缓存命中快速通道}: 当任务 $j$ 在RSU $r$ 缓存命中时 ($z_{j,r}=1$)
    \begin{itemize}
        \item 跳过数据上传、排队等待、计算处理阶段
        \item 直接进入结果下载流程,总时延为:
        \begin{equation}
            T_{total,j,r}^{\text{cache-hit}} = T_{\text{req\_up}, j, r} + T_{\text{trans\_down}, j, r}
        \end{equation}
        其中 $T_{\text{req\_up}, j, r}$ 为请求传输时延,$T_{\text{trans\_down}, j, r}$ 为结果下载时延
        \item 此机制显著降低系统时延和计算负载
    \end{itemize}
\end{itemize}

\textbf{排队时延预测模型}:
% 基于M/M/1非抢占式优先级队列理论的时延预测公式

\begin{itemize}
    \item \textbf{RSU排队时延预测}: 对于RSU $r$ 上优先级为 $p_j$ 的任务 $j$,当未发生缓存命中时:
    \begin{equation} \label{eq:T_wait_r_original}
        T_{wait, j, r}^{\text{pred}} \approx \frac{1}{\mu_r} \cdot \frac{\sum_{i=1}^{p_j} \rho_{i,r}}{(1 - \sum_{i=1}^{p_j-1} \rho_{i,r})(1 - \sum_{i=1}^{p_j} \rho_{i,r})}
    \end{equation}

    \item \textbf{参数定义}:
    \begin{itemize}
        \item $\lambda_{i,r}$: RSU $r$ 上优先级 $i$ 任务的平均到达率 (tasks/秒)
        \item $\mu_r = f_r/C_{avg,r}$: RSU $r$ 的平均服务速率 (tasks/秒)
        \item $C_{avg,r}$: RSU $r$ 处理任务的平均计算量 (CPU cycles)
        \item $\rho_{i,r} = \lambda_{i,r}/\mu_r$: 优先级 $i$ 任务的流量强度 (无量纲)
        \item 稳定性条件: $\sum_{i=1}^{P} \rho_{i,r} < 1$ (系统稳定运行的必要条件)
    \end{itemize}

    \item \textbf{UAV排队时延预测}: 对于UAV $u$ 上优先级为 $p_j$ 的任务 $j$:
    \begin{equation} \label{eq:T_wait_u_priority}
        T_{\text{wait},j,u}^{\text{pred}} \approx \frac{1}{\mu_u} \cdot \frac{\sum_{i=1}^{p_j} \rho_{i,u}}{(1 - \sum_{i=1}^{p_j-1} \rho_{i,u})(1 - \sum_{i=1}^{p_j} \rho_{i,u})}
    \end{equation}

    \item \textbf{UAV参数定义}:
    \begin{itemize}
        \item $\lambda_{i,u}$: UAV $u$ 上优先级 $i$ 任务的平均到达率 (tasks/秒)
        \item $\mu_u = f_u / C_{avg,u}$: UAV $u$ 的平均服务速率 (tasks/秒)
        \item $\rho_{i,u} = \lambda_{i,u}/\mu_u$: 优先级 $i$ 任务在UAV $u$ 上的流量强度
    \end{itemize}

    \item \textbf{车辆本地处理}: 车辆 $v_j$ 本地处理采用相同的优先级队列模型:
    \begin{equation}
        T_{wait,j,v_j}^{\text{pred}} \approx \frac{1}{\mu_{v_j}} \cdot \frac{\sum_{i=1}^{p_j} \rho_{i,v_j}}{(1 - \sum_{i=1}^{p_j-1} \rho_{i,v_j})(1 - \sum_{i=1}^{p_j} \rho_{i,v_j})}
    \end{equation}
\end{itemize}

\textbf{高负载预警}: 当系统负载接近饱和 ($\sum_{i=1}^{P} \rho_i \to 1$) 时,排队时延将急剧增长。实际系统需采用仿真方法或动态调整到达率 $\lambda$ 和服务率 $\mu$ 参数以获得精确的时延估算。

\subsection{决策变量与系统输入}
% VEC-RL系统的核心决策变量定义,用于强化学习智能体的动作空间构建

VEC-RL系统在每个时隙 $t$ 需要做出多种类型的决策,包括任务分配、资源调度、缓存管理等。以下定义主要决策变量和系统输入:

\begin{itemize}
    \item \textbf{任务分配决策变量}:
    \begin{itemize}
        \item $x_{j,n}^t \in \{0, 1\}$: 二进制任务分配决策 % 核心的任务卸载决策
        \begin{itemize}
            \item $x_{j,n}^t = 1$: 任务 $j$ 在时隙 $t$ 分配给节点 $n$ 处理
            \item $x_{j,n}^t = 0$: 任务 $j$ 不分配给节点 $n$
            \item 约束: $\sum_{n \in \mathcal{N}} x_{j,n}^t \leq 1$ (每个任务最多分配给一个节点)
        \end{itemize}
    \end{itemize}

    \item \textbf{数据流控制决策变量}:
    \begin{itemize}
        \item $D_{local,v,l,p}^t \geq 0$: 车辆 $v$ 本地处理数据量 (bits) % 车辆本地计算决策
        \item $D_{off,v,r,l,p}^t \geq 0$: 车辆 $v$ 卸载到RSU $r$ 的数据量 (bits) % V2I卸载决策
        \item $D_{off,v,u,l,p}^t \geq 0$: 车辆 $v$ 卸载到UAV $u$ 的数据量 (bits) % V2U卸载决策
        \item $D_{proc,r,l,p}^t \geq 0$: RSU $r$ 处理数据量 (bits) % RSU计算调度决策
        \item $D_{proc,u,l,p}^t \geq 0$: UAV $u$ 处理数据量 (bits) % UAV计算调度决策
        \item $D_{mig,r,r',l,p}^t \geq 0$: RSU间迁移数据量 (bits) % RSU间任务迁移决策
    \end{itemize}

    \item \textbf{资源管理决策变量}:
    \begin{itemize}
        \item $\mu_{r}^t \in \{0, 1\}$: RSU $r$ 计算单元激活状态 % RSU开关控制
        \item $\nu_{r,r'}^t \in \{0, 1\}$: RSU $r$ 向RSU $r'$ 的迁移决策 % 迁移开关控制
        \item $b_{link}^t \geq 0$: 通信链路带宽分配 (Hz) % 带宽资源分配
        \begin{itemize}
            \item $link \in \{(v,r), (v,u), (u,v), (r,r')\}$: 各类通信链路
            \item 约束: $\sum_{link} b_{link}^t \leq B_{total}$ (总带宽限制)
        \end{itemize}
    \end{itemize}

    \item \textbf{缓存管理决策变量}:
    \begin{itemize}
        \item $z_{j,r}^t \in \{0, 1\}$: RSU $r$ 对任务类型 $j$ 的缓存决策 % 缓存策略决策
        \begin{itemize}
            \item $z_{j,r}^t = 1$: 缓存任务 $j$ 的计算结果
            \item $z_{j,r}^t = 0$: 不缓存该结果
            \item 通常为较长周期的策略性决策
        \end{itemize}
    \end{itemize}

    \item \textbf{系统输入变量} (非决策变量):
    \begin{itemize}
        \item $D_{gen,v,l,p}^t \geq 0$: 车辆 $v$ 新生成任务数据量 (bits) % 外部任务到达
        \begin{itemize}
            \item 服从泊松过程,参数为 $\lambda_{v,l,p}$
            \item 直接进入车辆队列 $(l,p)$
            \item 为随机环境输入,不可控制
        \end{itemize}
    \end{itemize}
\end{itemize}

\textbf{决策序列与最终结果}: 最终的任务分配结果 $x_{j,n}$ (任务 $j$ 在节点 $n$ 完成处理) 是时隙序列 $\{x_{j,n}^t\}_{t=0}^{T}$ 决策的累积结果,体现了VEC-RL系统的动态决策过程。
\newpage
% --- 新增/保留的任务分类部分 ---
\section{基于延迟容忍度的任务分类与卸载导向}
% 根据QoS延迟要求对任务进行分类,指导智能卸载决策

VEC系统中的计算任务具有不同的延迟容忍度特征,需要采用分类策略来优化资源分配和卸载决策。

\subsection{任务分类框架}
% 基于延迟容忍度的四级任务分类体系

\textbf{分类原理}: 根据任务的最大可容忍延迟 $T_{max,j}$ 将任务划分为 $K=4$ 个类别,每个类别对应不同的QoS要求和卸载策略。

\textbf{延迟阈值定义}: 设置三个递增的延迟容忍度阈值 $\tau_1 < \tau_2 < \tau_3$ (单位: 时隙数),其具体数值可在配置文件 \texttt{vec\_system\_config.py} 中的 \texttt{TaskConfig} 类中设置:
\begin{itemize}
    \item $\tau_1 = $ \texttt{delay\_threshold\_1} (默认值: 2时隙)
    \item $\tau_2 = $ \texttt{delay\_threshold\_2} (默认值: 5时隙)
    \item $\tau_3 = $ \texttt{delay\_threshold\_3} (默认值: 10时隙)
\end{itemize}

\textbf{任务分类定义}: 任务 $j$ 的类别 $\text{Type}(j) = k$ 根据其延迟容忍度 $T_{max,j}$ 确定:
\begin{itemize}
    \item \textbf{类别 1} (极度延迟敏感型): $T_{\max,j} \leq \tau_1$
    \begin{itemize}
        \item 典型应用: 自动驾驶紧急制动、实时避障
        \item QoS要求: 超低延迟 ($\leq 2$ 时隙)
    \end{itemize}
    \item \textbf{类别 2} (延迟敏感型): $\tau_1 < T_{\max,j} \leq \tau_2$
    \begin{itemize}
        \item 典型应用: 实时导航、交通信号优化
        \item QoS要求: 低延迟 (2-5 时隙)
    \end{itemize}
    \item \textbf{类别 3} (中度延迟容忍型): $\tau_2 < T_{\max,j} \leq \tau_3$
    \begin{itemize}
        \item 典型应用: 视频处理、图像识别
        \item QoS要求: 中等延迟 (5-10 时隙)
    \end{itemize}
    \item \textbf{类别 4} (延迟容忍型): $T_{\max,j} > \tau_3$
    \begin{itemize}
        \item 典型应用: 数据分析、机器学习训练
        \item QoS要求: 高延迟容忍 ($> 10$ 时隙)
    \end{itemize}
\end{itemize}
\newpage
\subsection{分类驱动的卸载节点倾向性}
任务的类别 $Type(j)$ 决定了其优先考虑的候选计算节点集合 $\mathcal{N}_j^{cand} \subseteq \mathcal{N}$:
\begin{itemize}
    \item 类别 1 ($T_{\max,j} \leq \tau_1$): $\mathcal{N}_j^{cand} = \{v_j\}$。(仅考虑本地处理,因传输时延可能超过容忍度)。
    \item 类别 2 ($\tau_1 < T_{\max,j} \leq \tau_2$): $\mathcal{N}_j^{cand} \subseteq \{v_j\} \cup \mathcal{R}_{nearby, low-latency}$。(首选本地,次选极近低延迟 RSU)。
    \item 类别 3 ($\tau_2 < T_{\max,j} \leq \tau_3$): $\mathcal{N}_j^{cand} \subseteq \{v_j\} \cup \mathcal{R}_{reachable} \cup \mathcal{U}_{capable, nearby}$。(本地、可达 RSU、近距离且能力足够的 UAV)。
    \item 类别 4 ($T_{\max,j} > \tau_3$): $\mathcal{N}_j^{cand} = \mathcal{N}$。(所有节点皆可考虑,侧重成本效益)。
\end{itemize}
 $\mathcal{N}_j^{cand}$ 是一个初步筛选的候选集。最终决策 $x_{j,n}$ 必须在 $\mathcal{N}_j^{cand}$ 内(或结合缓存情况进行扩展),通过后续的详细评估和优化来确定,确保严格满足 $T_{max,j}$。 $\mathcal{R}_{nearby}, \mathcal{U}_{nearby}$ 等需要根据当前网络拓扑和预计传输时延动态判断。

\newpage

\section{处理模式评估框架 (在候选集 $\mathcal{N}_j^{cand}$ 内,并考虑缓存)}

对任务 $j$ (或对应的数据块),在其候选集 $\mathcal{N}_j^{cand}$ 内(以及对所有可达RSU检查缓存情况)评估处理模式。评估通常在每个时隙 $t$ 进行,核心是预测总完成时延 $T_{total,j}$ 并与 $T_{max,j} \times \Delta t$ 比较。

\subsection{模式一:本地计算 (评估节点 $n=v_j$)}
\textbf{前提}: $v_j \in \mathcal{N}_j^{cand}$。
\begin{itemize}
    \item 动作: 车辆 $n=v_j$ 决定使用本地 CPU 处理任务 $j$。
    \item 时延分析:
        \begin{itemize}
        \item 等待时延 ($T_{wait, j, v_j}$): 任务 $j$ 的数据在车辆 $v_j$ 的多优先级队列 $Q_{v_j}$ 中等待处理的时间。使用式 \eqref{eq:T_wait_r_original} 或 \eqref{eq:T_wait_u_priority} 的方法估计或基于当前队列积压进行瞬时估计。
        \item 计算时延 ($T_{comp, j, v_j}$): $T_{comp, j, v_j} = C_j / f_{v_j}$ 。
        \item 总时延预测: $T_{total, j, v_j}^{\text{pred}} = T_{wait, j, v_j} + T_{comp, j, v_j}$。
        \item 可行性检查: $T_{total, j, v_j}^{\text{pred}} \leq T_{max,j} \times \Delta t$
        \end{itemize}
    \item 队列交互: 若决策执行,数据 $D_j$ 从 $q_{v_j,l,p}^t$ (任务 $j$ 所在队列) 中移除。
\end{itemize}

\subsection{模式二:卸载到 RSU (评估 $r \in \mathcal{R}$,优先考虑 $\mathcal{N}_j^{cand}$ 内的RSU)}
\text{前提}: 车辆 $n=v_j$ 在 RSU $r$ 的通信范围内。
\begin{itemize}
    \item \textbf{步骤1: 检查RSU $r$ 是否缓存了任务 $j$ 的结果 ( $z_{j,r}=1$)}。
    \item \textbf{如果 $z_{j,r}=1$ (缓存命中):}
        \begin{itemize}
            \item 动作: 车辆 $n=v_j$ 从 RSU $r$ 请求已缓存的结果 $S_j$。
            \item 请求上传时延 ($T_{\text{req\_up},j,r}$): 车辆发送一个小的请求包给RSU $r$ 的时延。$T_{\text{req\_up},j,r} = D_{req} / R_{n,r}$,其中 $D_{req}$ 是请求包大小 (通常很小)。
            \item 结果下载时延 ($T_{\text{trans\_down},j,r}$): $T_{\text{trans\_down},j,r} = S_j / R_{r,n}$。
            \item 总时延预测 (缓存命中): $T_{total,j,r}^{\text{pred, cache-hit}} = T_{\text{req\_up},j,r} + T_{\text{trans\_down},j,r}$。
            \item 可行性检查: $T_{total,j,r}^{\text{pred, cache-hit}} \leq T_{max,j} \times \Delta t$。
            \item 队列交互: 任务 $j$ 不进入 RSU $r$ 的计算队列 $G_r$。原始数据 $D_j$ 无需从车辆上传到RSU。车辆队列 $q_{v_j,l,p}^t$ 中对应任务 $j$ 的数据被移除(或标记为已处理)。
        \end{itemize}
    \item \textbf{如果 $z_{j,r}=0$ (缓存未命中) 并且 $r \in \mathcal{N}_j^{cand}$ :}
        \begin{itemize}
            \item 动作: 车辆 $n=v_j$ 决定卸载原始任务数据 $D_j$ 到 RSU $r$ 进行处理。
            \item 时延分析:
                \begin{itemize}
                    \item 上传传输时延 ($T_{trans\_up, j, r}$): $T_{trans\_up, j, r} = D_j/R_{n,r}$。
                    \item 等待时延 ($T_{wait, j, r}$): 任务 $j$ 在 RSU $r$ 的多优先级队列 $G_r$ 中等待处理的时间。\\
                    \textbf{长期平均预测:} 使用式~\eqref{eq:T_wait_r_original}。\\
                                        \item 瞬时积压预测 (用于MDP状态,任务$j$优先级为$p_j$,在$t_{arr}$时刻到达RSU $r$):
                        \begin{equation} \label{eq:T_wait_r_instantaneous_pred_sec4} % Ensure unique label
                            T_{wait, j, r}^{\text{inst-pred}}(t_{arr}) \approx \frac{C_{rem,r}(t_{arr}) + \sum_{p'=1}^{p_j-1} \left( \left( \sum_{l'=1}^{L-1} g_{r,l',p'}(t_{arr}) \right) \cdot c \right)}{f_r}
                        \end{equation}
                        其中 $C_{rem,r}(t_{arr})$ 是 $t_{arr}$ 时刻RSU $r$ 上正在服务任务的剩余计算量 (若CPU空闲则为0),$g_{r,l',p'}(t_{arr})$ 是 $t_{arr}$ 时刻RSU $r$ 队列 $(l',p')$ 中的数据量,$c$ 是处理密度,$f_r$ 是RSU $r$ 的计算能力。该公式估算在任务 $j$ 前面所有更高优先级任务及当前任务(若有)的处理时间。
                    \item 计算时延 ($T_{comp, j, r}$): $T_{comp, j, r} = C_j/f_r$。
                    \item 下载传输时延 ($T_{trans\_down, j, r}$): $T_{trans\_down, j, r} = S_j/R_{r,n}$。
                    \item 总时延预测 (缓存未命中): $T_{total, j, r}^{\text{pred, no-cache}} = T_{trans\_up, j, r} + T_{wait, j, r} + T_{comp, j, r} + T_{trans\_down, j, r}$。
                \end{itemize}
            \item 可行性检查: $T_{total, j, r}^{\text{pred, no-cache}} \leq T_{max,j} \times \Delta t$。
            \item 队列交互: 数据 $D_j$ 从车辆 $v_j$ 的 $q_{v_j,l,p}^t$ 移除。假设传输消耗一个时隙,任务将加入 RSU $r$ 的 $g_{r,l-1,p}$ 队列 (如果 $l>1$) 或一个特殊的到达队列。处理后从相应RSU队列移除。
        \end{itemize}
\end{itemize}

\subsection{模式三:RSU 间迁移 (评估 $k \to k'$ for $k, k' \in \mathcal{R}$)}
\textbf{前提}: RSU $k$ 存有待处理数据(该数据未在$k'$缓存),迁移到 RSU $k'$ 处理可能更优。
\begin{itemize}
    \item 动作: RSU $k$ 迁移某任务对应的数据块 (例如 $D_j$) 到 RSU $k'$。
    \item 迁移时延 ($T_{mig, k, k'}$): $T_{mig, k, k'}(D_j) = D_j / R_{k,k'}$ (见 Eq. \ref{eq:T_trans_mig_revised})。
    \item 目的: 优化任务整体端到端时延。评估需考虑迁移时延及后续在 $k'$ 的等待和计算时延(假设 $k'$ 未缓存结果)。
    \item 队列交互: 数据 $D_{mig,k,k',l,p}^t$ 从 $g_{k,l,p}^t$ 移除。假设迁移消耗一个时隙,数据将加入 $g_{k',l-1,p}$ 队列 (如果 $l>1$) 或一个特殊到达队列。
\end{itemize}

\subsection{模式四:卸载到 UAV (评估 $u \in \mathcal{U} \cap \mathcal{N}_j^{cand}$)}
\textbf{前提}: 候选 UAV $u \in \mathcal{N}_j^{cand}$,且车辆 $n=v_j$ 在 $u$ 的通信范围内。 (UAV $u$ 位置固定)
\begin{itemize}
    \item 动作: 车辆 $n=v_j$ 决定卸载任务 $j$ 到 UAV $u$。
    \item 时延分析 (预测):
        \begin{itemize}
            \item 上传传输时延: $T_{\text{trans\_up},j,u} = D_j / R_{n,u}$。
            \item 等待时延 (优先级队列): $T_{\text{wait},j,u}^{\text{pred}}$ 使用式 \eqref{eq:T_wait_u_priority} (长期平均) 或式(4) 的瞬时积压估计。
            \item 执行时延: $T_{\text{exec},j,u} = C_j / f_u$。
            \item 下载传输时延: $T_{\text{trans\_down},j,u} = S_j / R_{u,n}$。
            \item 总时延预测: $T_{\text{total},j,u}^{\text{pred}} = T_{\text{trans\_up},j,u} + T_{\text{wait},j,u}^{\text{pred}} + T_{\text{exec},j,u} + T_{\text{trans\_down},j,u}$。(参考 Eq. \ref{eq:T_total_u_revised})。
        \end{itemize}
    \item 可行性检查: $T_{\text{total},j,u}^{\text{pred}} \leq T_{max,j} \times \Delta t$
    \item 队列交互: 数据 $D_j$ 从车辆 $v_j$ 的 $q_{v_j,l,p}^t$ 移除。假设传输消耗一个时隙,任务将加入 UAV $u$ 的 $u_{u,l-1,p}$ 队列 (如果 $l>1$) 或一个特殊到达队列,处理后从相应UAV队列移除。
\end{itemize}

\newpage
\section{通信与计算模型 (公式汇总)}

\subsection{本地计算模型 (车辆 $n \in \mathcal{V}$)}
% 基于实际车载计算单元(如NVIDIA Jetson, Intel NUC)的能耗特性建模

\begin{align}
    \text{处理能力 (数据量/时隙)}: \quad & D^{local}_{n} = \frac{f_{n} \Delta t}{c} \label{eq:D_local_n_revised_sec5} \\
    %
    \text{计算时延 (任务 j)}: \quad & T_{comp, j, n} = \frac{C_j}{f_n \cdot \eta_{parallel}} = \frac{c D_j}{f_n \cdot \eta_{parallel}} \label{eq:T_comp_revised_sec5} \\
    %
    \text{动态功率模型}: \quad & P^{comp}_{n}(f_n, U_n) = \kappa_1 f_{n}^3 + \kappa_2 f_{n}^2 U_n + P_{static} \label{eq:P_comp_n_revised_sec5} \\
    %
    \text{处理能耗 (每时隙)}: \quad & E^{comp}_{n,t} = P^{comp}_{n}(f_n, U_{n,t}) \cdot \tau_{active,n,t} + P_{idle} \cdot (\Delta t - \tau_{active,n,t}) \label{eq:E_comp_n_revised_sec5}
\end{align}

\textbf{参数定义} (基于实际硬件数据):
\begin{itemize}
    \item $\eta_{parallel}$: 并行处理效率因子 (0.7-0.9) % 考虑多核并行效率损失
    \item $U_{n,t}$: 时隙 $t$ 内CPU利用率 (0-1) % 实际负载情况
    \item $\kappa_1$: 频率立方项系数 % 基于CMOS工艺特性
    \begin{itemize}
        \item 车载ARM处理器: $1.0 \times 10^{-28}$ W·s³/cycle³
        \item 车载x86处理器: $2.5 \times 10^{-28}$ W·s³/cycle³
    \end{itemize}
    \item $\kappa_2$: 频率平方项系数 (考虑负载相关功耗)
    \begin{itemize}
        \item 典型值: $5.0 \times 10^{-19}$ W·s²/cycle²
    \end{itemize}
    \item $P_{static}$: 静态功耗 (漏电流等)
    \begin{itemize}
        \item 车载处理器: 2-8W (取决于工艺节点)
    \end{itemize}
    \item $P_{idle}$: 空闲功耗
    \begin{itemize}
        \item 典型值: $P_{static} + 0.1 \times P_{max}$
    \end{itemize}
    \item $\tau_{active,n,t}$: 实际计算活动时长 $\leq \Delta t$
\end{itemize}

\textbf{实际参考值} (可在配置文件中调整):
\begin{itemize}
    \item NVIDIA Jetson Xavier NX: $f_{max} = 1.9$ GHz, $P_{max} = 20$ W
    \item Intel NUC i7: $f_{max} = 4.2$ GHz, $P_{max} = 65$ W
    \item Qualcomm Snapdragon 8cx: $f_{max} = 3.0$ GHz, $P_{max} = 15$ W
\end{itemize}

\subsection{无线通信模型 (链路 $a \to b$, e.g., V2I, V2U, U2V)}
% 基于3GPP标准的VEC无线通信信道模型,考虑实际传播环境特性

假设节点 $a$ 向节点 $b$ 传输数据 $D$,分配带宽为 $B_{a,b}^t$ (由决策变量确定)。

\begin{align}
    \text{节点间距离}: \quad & d_{a,b}(t) = \lVert \text{pos}_a(t) - \text{pos}_b(t) \rVert \quad (\text{若 } a \text{ 或 } b \text{ 是UAV, 其位置固定}) \\
    %
    \text{视距概率模型}: \quad & P_{\text{LoS},a,b}(t) = \begin{cases}
        1, & \text{if } d_{a,b}(t) \leq d_0 \\
        \exp\left(-\frac{d_{a,b}(t) - d_0}{\alpha_{LoS}}\right), & \text{if } d_{a,b}(t) > d_0
    \end{cases} \label{eq:LoS_prob} \\
    %
    \text{路径损耗模型 (dB)}: & \nonumber \\
    & L_{\text{LoS}}(d) = 32.4 + 20\log_{10}(f_c) + 20\log_{10}(d) + X_{\sigma, \text{LoS}} \label{eq:path_loss_LoS} \\
    & L_{\text{NLoS}}(d) = 32.4 + 20\log_{10}(f_c) + 30\log_{10}(d) + X_{\sigma, \text{NLoS}} \label{eq:path_loss_NLoS} \\
    %
    \text{综合路径损耗}: \quad & L_{a,b}(t) = P_{\text{LoS},a,b}(t) L_{\text{LoS}}(d_{a,b}(t)) + (1 - P_{\text{LoS},a,b}(t)) L_{\text{NLoS}}(d_{a,b}(t)) \\
    %
    \text{信道增益 (线性)}: \quad & h_{a,b}(t) = 10^{-L_{a,b}(t)/10} \cdot g_{antenna} \cdot g_{fading}(t) \label{eq:channel_gain} \\
    %
    \text{干扰功率}: \quad & I_{ext,b}(t) = \sum_{\substack{i \neq a \\ \text{同频段}}} P_{tx,i} h_{i,b}(t) + I_{thermal} \label{eq:interference} \\
    %
    \text{信噪干扰比}: \quad & \text{SINR}_{a,b}(t) = \frac{P_{tx,a} h_{a,b}(t)}{I_{ext,b}(t) + N_0 B_{a,b}^t} \label{eq:SINR_revised_sec5}\\
    %
    \text{传输速率 (Shannon)}: \quad & R_{a,b}(t) = B_{a,b}^t \log_2 (1 + \text{SINR}_{a,b}(t)) \cdot \eta_{coding} \label{eq:Rate_revised_sec5} \\
    %
    \text{传输时延}: \quad & T_{trans, a, b}(D,t) = \frac{D}{R_{a,b}(t)} + T_{prop}(d_{a,b}(t)) + T_{proc} \label{eq:T_trans_revised_sec5} \\
    %
    \text{传输量 (每时隙)}: \quad & D^{trans}_{a,b}(t) = R_{a,b}(t) \Delta t \label{eq:D_trans_revised_sec5} \\
    %
    \text{传输能耗}: \quad & E^{tx}_{a,t} = P_{tx,a} \cdot \tau_{tx,a,t} + P_{circuit} \cdot \tau_{active,a,t} \label{eq:E_trans_revised_sec5}
\end{align}

\textbf{参数定义} (可在配置文件中设置):
\begin{itemize}
    \item $f_c$: 载波频率 (Hz) - \texttt{config.communication.carrier\_frequency}
    \item $d_0$: 视距临界距离 (m) - 通常为50-100m
    \item $\alpha_{LoS}$: 视距概率衰减因子 - 典型值200-500m
    \item $X_{\sigma, \text{LoS/NLoS}}$: 阴影衰落 (dB) - 对数正态分布,标准差3-8dB
    \item $g_{antenna}$: 天线增益 - 典型值0-10dBi
    \item $g_{fading}(t)$: 快衰落因子 - 瑞利/莱斯分布
    \item $\eta_{coding}$: 编码效率 - 典型值0.8-0.95
    \item $T_{prop}(d)$: 传播时延 $= d/c$ (光速)
    \item $T_{proc}$: 处理时延 - 典型值1-5ms
    \item $P_{circuit}$: 电路功耗 - 典型值50-200mW
    \item $I_{thermal}$: 热噪声功率 $= k_B T_0 B$
\end{itemize}

\subsection{RSU 计算模型 (RSU $k \in \mathcal{R}$)}
\begin{align}
    \text{处理能力 (数据量/时隙)}: \quad & D^{proc}_{k} = \frac{f_{k} \Delta t}{c} \label{eq:D_proc_k_revised_sec5} \\ 
    \text{计算时延 (任务 j, 秒)}: \quad & T_{comp, j, k} = \frac{C_j}{f_k} = \frac{c D_j}{f_k} \label{eq:T_comp_k_revised_sec5} \\ 
    \text{处理功率}: \quad & P^{comp}_{k} = \kappa_2 (f_{k})^3 \label{eq:P_comp_k_revised_sec5} \\ 
    \text{处理能耗 (每时隙, 若激活)}: \quad & E^{comp}_{k,t} = \mu_k^t P^{comp}_{k} \cdot \tau_{active,k,t} \label{eq:E_comp_k_revised_sec5} 
\end{align}
其中 $\kappa_2$ 是 RSU 芯片相关的有效电容系数,$\tau_{active,k,t}$ 是 RSU $k$ 在时隙 $t$ CPU 的实际活动时长。

\subsection{RSU 间迁移通信模型 (链路 $k \to k'$, RSU $k$ 发送)}
采用与5.2节类似的无线通信模型,但可能使用不同的参数(如专用回程链路)。
\begin{align}
    \text{传输速率 (bps)}: \quad & R_{k,k'}(t) = B_{k,k'}^t \log_2 (1 + \text{SINR}_{k,k'}(t)) \label{eq:Rate_mig_revised_sec5} \\ 
    \text{迁移传输时延 (数据 D, 秒)}: \quad & T_{trans, k, k'}(D,t) = \frac{D}{R_{k,k'}(t)} \label{eq:T_trans_mig_revised} \\
    \text{传输量 (bits/时隙)}: \quad & D^{mig}_{k,k'}(t) = R_{k,k'}(t) \Delta t \label{eq:D_mig_revised_sec5} \\ 
    \text{传输能耗 (RSU $k$, 每时隙, 若迁移)}: \quad & E^{tx,mig}_{k,t} = \nu_{k,k'}^t P_{tx,k}^{mig} \cdot \tau_{tx,k,t}^{mig} \label{eq:E_mig_revised_sec5} 
\end{align}
$P_{tx,k}^{mig}$ 为 RSU 间迁移的发射功率,$\tau_{tx,k,t}^{mig}$ 为实际迁移发射时长。

\subsection{UAV 计算与排队模型 (UAV $u \in \mathcal{U}$, 多优先级 M/M/1 类型预测)}
针对任务 $j$ (大小 $D_j$, 复杂度 $C_j=c D_j$, 优先级 $p_j$) 进行预测。
\begin{align}
    \text{UAV 平均服务速率 (tasks/sec)}: \quad & \mu_u = f_u / C_{avg,u} \label{eq:mu_u_revised_sec5} \\ 
    \text{UAV 优先级 $i$ 任务平均到达率 (tasks/sec)}: \quad & \lambda_{i,u} = (\sum_{j': \text{assigned to } u, \text{Pri}(j')=i} \lambda_{j'}') / \Delta t \quad (\text{动态估计}) \label{eq:lambda_iu_revised_sec5} \\ 
    \text{UAV 优先级 $i$ 系统负载}: \quad & \rho_{i,u} = \lambda_{i,u} / \mu_u \quad (\text{需确保 } \sum_i \rho_{i,u} < 1) \label{eq:rho_iu_revised_sec5} \\ 
    \text{平均排队等待时延 (秒, 优先级 $p_j$)}: \quad & T_{\text{wait},j,u}^{\text{pred}} \text{ (见式\ref{eq:T_wait_u_priority})} \label{eq:T_wait_u_revised_sec5} \\ 
    \text{任务 $j$ 计算时延 (秒)}: \quad & T_{\text{exec},j,u} = C_j / f_u = c D_j / f_u \label{eq:T_exec_u_revised_sec5} \\ 
    \text{任务 $j$ 总时延预测 (秒)}: \quad & T_{\text{total},j,u}^{\text{pred}} = T_{trans, n, u}(D_j,t) + T_{\text{wait},j,u}^{\text{pred}} + T_{\text{exec},j,u} + T_{trans, u, n}(S_j,t) \label{eq:T_total_u_revised} \\
     & = (D_j / R_{n,u}(t)) + T_{\text{wait},j,u}^{\text{pred}} + (C_j / f_u) + (S_j / R_{u,n}(t)) \nonumber \\
    \text{任务 $j$ 计算能耗 (焦耳)}: \quad & E^{comp}_{u,j} = \kappa_3 f_u^2 C_j = \kappa_3 f_u^2 c D_j \label{eq:E_comp_u_revised_sec5} \\ 
    \text{时隙 $t$ 内UAV $u$ 计算总能耗}: \quad & E^{comp}_{u,t} = \kappa_3 (f_u(t))^3 \cdot \tau_{active,u,t} \quad \text{或基于处理任务的累积} \label{eq:E_comp_ut_revised_sec5} 
\end{align}
其中 $C_{avg,u}$ 是 UAV 平均任务复杂度,$\kappa_3$ 是 UAV 计算能耗系数。$f_u(t)$ 为时隙 $t$ UAV 的计算频率 (如果可调)。

\subsection{UAV 特定能耗模型} \label{sec:uav_energy_revised_sec5}

\subsubsection{UAV 通信能耗}
UAV 通信能耗包括接收任务数据和发送结果数据。
\begin{itemize}
    \item 接收能耗 (任务 $j$):
        \[ E^{rx}_{u,j}(t) = P_{rx,u} \times T_{trans, n, u}(D_j,t) = P_{rx,u} \frac{D_j}{R_{n,u}(t)} \]
    \item 发送能耗 (任务 $j$):\[ E^{tx}_{u,j}(t) = P_{tx,u} \times T_{trans, u, n}(S_j,t) = P_{tx,u} \frac{S_j}{R_{u,n}(t)} \]
    \item 任务 $j$ 的总通信能耗 (UAV 侧):\[ E^{comm}_{u,j}(t) = E^{rx}_{u,j}(t) + E^{tx}_{u,j}(t) \]
  \item 时隙 $t$ 内的通信总能耗 $E^{comm,t}_u$:
        \[ E^{comm,t}_u = P_{rx,u} \cdot \tau_{rx,u,t} + P_{tx,u} \cdot \tau_{tx,u,t} \]
        其中 $\tau_{rx,u,t}, \tau_{tx,u,t}$ 分别表示时隙 $t$ 内UAV $u$ 的实际接收/发送时长。
\end{itemize}

\subsubsection{UAV 飞行能耗}
%% MODIFIED: UAV is fixed, so flight energy is constant hovering energy.
由于 UAV 位置固定,其飞行能耗为保持悬停状态的能耗。

\textbf{完整的UAV飞行功率公式}(简化前):
\begin{equation} \label{eq:P_fly_revised_sec5}
    P_{fly,u}(v) = P_0 + P_i \frac{1}{v^2} + \frac{1}{2} \rho A C_d v^3
\end{equation}
其中 $v$ 为飞行速度,在悬停状态下 $v=0$,公式简化为悬停功率。

\begin{align}
    \text{悬停功率}: \quad & P_{hover,u} = P_0 + P_i \label{eq:P_hover_revised_sec5} \\
    \text{时隙 } t \text{ 内的悬停能耗}: \quad & E^{fly,t}_u = P_{hover,u} \Delta t \label{eq:E_fly_t_revised_sec5}
\end{align}
其中 $P_0$ 是悬停时的叶型功率系数,$P_i$ 是悬停时的诱导功率系数。原先与速度相关的飞行功率公式 \eqref{eq:P_fly_revised_sec5} 在此简化。


\newpage
%% REMOVED BY USER REQUEST: Section 8 (Vehicle Trajectory Prediction) is removed.
\iffalse 
\newpage % Start a new page for clarity

\section{优化问题定式}
\label{sec:optimization_problem}
% 本节将前面定义的系统模型、决策变量和性能指标整合为一个统一的优化问题框架。

基于前述的系统模型,我们的核心目标是在满足各类服务质量(QoS)和物理资源约束的前提下,联合优化系统的**总时延**和**总能耗**。这是一个典型的多目标优化问题,我们通过加权和(weighted sum)方法将其转化为一个单目标优化问题。

\subsection{目标函数}
我们旨在最小化系统在一段时间内的加权总成本,该成本由任务完成时延、系统总能耗和任务丢弃惩罚三部分组成。对于单个时隙 $t$ 的决策,优化目标可以表述为:

\begin{equation}
\label{eq:objective_function}
\min_{\mathbf{X}^t, \mathbf{Z}^t, \mathbf{B}^t, \ldots} \quad w_T \sum_{j \in \mathcal{J}_t} \sum_{n \in \mathcal{N}} x_{j,n}^t T_{total, j, n}^{\text{pred}} + w_E E_{total}^t + w_P \sum_{j \in \mathcal{J}_t} \left(1 - \sum_{n \in \mathcal{N}} x_{j,n}^t\right) P_{drop,j}
\end{equation}

其中:
\begin{itemize}
    \item $\mathbf{X}^t = \{x_{j,n}^t\}$, $\mathbf{Z}^t = \{z_{j,r}^t\}$, $\mathbf{B}^t = \{b_{link}^t\}$ 等代表在时隙 $t$ 的所有决策变量集合。
    \item $w_T, w_E, w_P$ 是用于平衡时延、能耗和任务成功率三个目标的非负权重系数,其和通常归一化为1 ($w_T + w_E + w_P = 1$)。这些权重可以根据不同的应用场景(例如,高续航场景 vs. 低延迟场景)进行调整。
    \item $T_{total, j, n}^{\text{pred}}$ 是将任务 $j$ 分配给节点 $n$ 后的**预测总完成时延**,其计算方式已在第4节"处理模式评估框架"和第5节"通信与计算模型"中详细定义(例如,式 \ref{eq:T_total_u_revised})。
    \item $E_{total}^t$ 是系统在时隙 $t$ 的**总能耗**,由式 \eqref{eq:E_total_t_final_revised_sec7} 给出,它是所有决策变量的函数。
    \item $(1 - \sum_{n \in \mathcal{N}} x_{j,n}^t)$ 是一个指示函数,当任务 $j$ 在时隙 $t$ 未被分配给任何节点(即被丢弃)时,其值为1,否则为0。
    \item $P_{drop,j}$ 是丢弃任务 $j$ 的惩罚成本,可以是一个较大的常数,也可以与任务的优先级或类型相关,以优先保证高优先级任务的完成。
\end{itemize}

\subsection{决策变量}
优化问题求解的是在每个决策周期(例如,时隙 $t$)内最优的决策变量集,主要包括:
\begin{itemize}
    \item \textbf{任务卸载与分配决策}: $\mathbf{X}^t = \{x_{j,n}^t \mid \forall j \in \mathcal{J}_t, n \in \mathcal{N}\}$。
    \item \textbf{缓存决策}: $\mathbf{Z}^t = \{z_{j,r}^t \mid \forall j \in \mathcal{J}, r \in \mathcal{R}\}$ (通常是较长周期的策略决策)。
    \item \textbf{带宽分配决策}: $\mathbf{B}^t = \{b_{link}^t\}$,为所有活跃的通信链路分配带宽。
    \item \textbf{节点激活与迁移决策}: $\{\mu_k^t\}, \{\nu_{k,k'}^t\}$。
\end{itemize}

\subsection{约束条件}
优化过程必须满足在第5节末尾定义的所有系统约束条件,包括:
\begin{itemize}
    \item \textbf{硬约束}: 任务唯一分配 \eqref{eq:task_unique}、队列稳定性 \eqref{eq:queue_stability}、带宽容量 \eqref{eq:bandwidth_up}-\eqref{eq:bandwidth_down}、缓存容量 \eqref{eq:cache_capacity}、队列容量 \eqref{eq:queue_capacity_hard}、能耗预算 \eqref{eq:energy_budget}、数据流守恒 \eqref{eq:flow_conservation}、处理能力 \eqref{eq:processing_capacity} 以及变量域约束 \eqref{eq:binary_vars}-\eqref{eq:non_negative_vars}。
    \item \textbf{软约束}: 任务延迟QoS \eqref{eq:delay_soft}、通信质量 \eqref{eq:sinr_soft} 和负载均衡 \eqref{eq:load_balance}。这些软约束通常通过在目标函数中添加惩罚项来处理,而不是作为严格的限制。我们定义的目标函数中的任务丢弃惩罚项就是一种处理软约束的方式。
\end{itemize}

\subsection{问题性质}
该优化问题是一个大规模的**混合整数非线性规划(Mixed-Integer Non-Linear Programming, MINLP)**问题。其中,任务分配和缓存决策是整数(二元)变量,而带宽分配是连续变量。目标函数和部分约束(如传输速率与SINR的关系)是非线性的。

由于该问题的NP-hard性质和VEC环境的动态性(任务随机到达、信道时变),传统的优化求解器难以在短时间内找到最优解。这为采用**深度强化学习(DRL)**等启发式或学习型方法提供了充分的理由,DRL智能体通过与环境交互来学习一个近似最优的策略,以动态地做出决策来最小化长期累积成本。
\newpage
\section{车辆轨迹预测}
\label{sec:trajectory_prediction}

车辆轨迹预测对于边缘计算中的资源分配、任务卸载和UAV路径规划至关重要。

\subsection{模块目标}
\begin{itemize}
    \item 预测未来 $H_{pred}$ 个时隙内,每辆车 $v \in \mathcal{V}$ 的位置序列:
    \[ \text{PredPos}_v(t+1), \text{PredPos}_v(t+2), \dots, \text{PredPos}_v(t+H_{pred}) \]
\end{itemize}

\subsection{输入特征}
对于车辆 $v$ 在当前时隙 $t$,可用的输入特征包括:
\begin{itemize}
    \item \textbf{历史轨迹数据} (过去 $W_h$ 个时隙):
    \begin{itemize}
        \item 位置序列: $\{ \text{pos}_v(t-i) \}_{i=0}^{W_h-1}$
        \item 速度序列: $\{ \text{speed}_v(t-i) \}_{i=0}^{W_h-1}$
    \end{itemize}
     \item \textbf{道路网络信息} (若可用):
    \begin{itemize}
        \item 车辆 $v$ 当前所在路段的属性。
        \item 前方路网拓扑信息。
    \end{itemize}
    \item \textbf{其他上下文信息} (若可用): 交通状况、驾驶员意图(模拟中)。
\end{itemize}
 $\mathbf{F}_{v,t}^{traj}$ 表示用于轨迹预测的车辆 $v$ 在时隙 $t$ 的综合输入特征向量。

\subsection{预测模型}
\begin{itemize}
    \item \textbf{Seq2Seq},
    利用编码器-解码器架构处理时序依赖性。
    \begin{align*}
        \text{ContextVector}_v & = \text{Encoder}(\mathbf{F}_{v,t-W_h+1}^{traj}, \dots, \mathbf{F}_{v,t}^{traj}) \\
        \{\text{PredPos}_v(t+h)\}_{h=1}^{H_{pred}} & = \text{DecoderLSTM}(\text{ContextVector}_v)
    \end{align*}
\end{itemize}

\subsection{损失函数}
常用的损失函数是均方误差 (MSE):
\begin{equation}
    \mathcal{L}_{traj} = \frac{1}{|\mathcal{V}_{pred}| H_{pred}} \sum_{v \in \mathcal{V}_{pred}} \sum_{h=1}^{H_{pred}} || \text{PredPos}_v(t+h) - \text{pos}_v(t+h) ||_2^2
\end{equation}
其中 $\mathcal{V}_{pred}$ 是被预测的车辆集合,$\text{pos}_v(t+h)$ 是车辆 $v$ 在未来时隙 $t+h$ 的真实位置。

\subsection{模块集成}
在每个DRL决策时隙 $t$ 开始时,调用此模块:
\[ \{\text{PredPos}_v(t+1), \dots, \text{PredPos}_v(t+H_{pred})\}_{v \in \mathcal{V}} = \text{TrajectoryPredictor}(\{\mathbf{F}_{v,t}^{traj}\}_{v \in \mathcal{V}}) \]
预测结果将作为DRL Agent状态的一部分。
\fi

\section{任务迁移模型与低中断切换}

\subsection{迁移类型与触发机制}
本文实现五类迁移:1) RSU$\rightarrow$RSU,2) RSU$\rightarrow$UAV,3) UAV$\rightarrow$RSU,4) 车辆跟随迁移(Vehicle-Follow),5) 预防性迁移(Preemptive)。当源节点负载超过阈值、UAV 电量不足或车辆移动导致服务质量下降时触发迁移。为降低中断时间(downtime),采用 Keep-Before-Break(KBB)思想(在实现中通过冷却窗口、迁移超时、目标预热/缓存预取与带宽保底等机制近似实现)。

\subsection{目标选择与评分函数}
设源节点为 $s$、候选目标为 $t$,综合评分定义为:
\begin{equation}
  \mathrm{score}(s,t) = w_{\mathrm{load}} (1-\mathrm{load}_t) + w_{\mathrm{dist}} \Big(1-\min\big(\tfrac{d(s,t)}{d_{\max}},1\big)\Big),
\end{equation}
其中 $w_{\mathrm{load}}=0.6,\ w_{\mathrm{dist}}=0.4$,$d_{\max}$ 为最大距离阈值(可配)。最终选择 $t^*=\arg\max_t \mathrm{score}(s,t)$,并满足 UAV 电量、RSU/UAV 过载与距离等可行性约束。

\subsection{迁移成本、时延与成功概率}
对任务 $i$,数据量 $D_i$(MB)、计算量 $C_i$(cycles)、链路带宽 $B$(Mbps):
\begin{align}
  T_{\mathrm{tx}} &= \frac{D_i}{B}, \quad
  T_{\mathrm{net}} = \tau_0 + \kappa \cdot d(s,t), \quad
  T_{\mathrm{proc}} = \beta \cdot C_i, \\
  T_{\mathrm{mig}} &= T_{\mathrm{tx}} + T_{\mathrm{net}} + T_{\mathrm{proc}},
\end{align}
迁移成本:
\begin{equation}
  C_{\mathrm{mig}} = \alpha_{\mathrm{comp}} C_i + \alpha_{\mathrm{tx}} T_{\mathrm{tx}} + \alpha_{\mathrm{lat}} T_{\mathrm{net}}.
\end{equation}
成功概率(实现为基于负载、距离与任务复杂度的启发式):
\begin{equation}
  p_{\mathrm{succ}} = \min\!\Big(0.99,\ \max\!\big(0.1,\ p_0 f(\mathrm{load}_s) g(\mathrm{load}_t) h(d(s,t)) q(\mathrm{complexity}_i)\big)\Big).
\end{equation}

\subsection{低中断切换思路(工程实现)}
在工程上将切换拆分为"异步准备 + 短暂收敛"两阶段:
\begin{enumerate}
  \item \textbf{目标预热与资源预留}:在目标侧预留 CPU/内存/带宽,预启动运行时并预取热点数据(通过缓存管理器实现)。
  \item \textbf{增量同步/小粒度传输}:根据链路质量自适应分块与压缩传输任务状态(在现实现中以迁移成本/时延建模近似)。
  \item \textbf{短静默与句柄切换}:在短窗口内完成最终差异同步与句柄切换(downtime 近似由迁移时延与排队时延中的同步段给出)。
  \item \textbf{失败回滚与冷却}:迁移失败立即回滚并进入冷却窗口,避免频繁抖动。
\end{enumerate}

\subsection{迁移紧急度与决策}
设源/目标负载为 $L_s,L_t$,任务优先级为 $\mathrm{prio}(i)$,车辆移动导致的紧迫度为 $\mathrm{mobility}(v)$:
\begin{equation}
  U = \gamma_u (L_s - L_t) + \gamma_p\, \mathrm{prio}(i) + \gamma_m\, \mathrm{mobility}(v),
\end{equation}
最终决策最大化 $U - \lambda C_{\mathrm{mig}}$,并受资源、距离、电量、冷却期与超时等约束(均在配置中可调)。

\subsection{伪代码}
\begin{algorithm}[H]
\caption{低中断任务迁移(KBB 思想 + 工程近似)}
\Input{任务 $i$,源 $s$,候选集 $\mathcal{T}$,配置}
\Output{迁移结果}
$t^* \leftarrow \arg\max_{t\in \mathcal{T}} \mathrm{score}(s,t)$;\\
\If{not feasible($t^*$)}{\textbf{return} FAIL}
pre\_attach($t^*$, reserve\_bw, warm\_env);\\
\While{time\_left() \&\& not converged}{
  transmit\_delta(adaptive\_chunk, compress);\\
}
short\_quiesce();\\
switch\_handles($s \rightarrow t^*$);\\
cleanup($s$);\\
\textbf{return} SUCCESS
\end{algorithm}

\section{边缘缓存模型与策略实现}

\subsection{热度建模与Zipf流行度}
结合历史与时间槽热度并引入衰减:
\begin{align}
  H_{\mathrm{hist}}(c) &\leftarrow \rho\, H_{\mathrm{hist}}(c) + w_{\mathrm{app}}(c), \\
  H_{\mathrm{slot}}(c,t) &\leftarrow H_{\mathrm{slot}}(c,t) + w_{\mathrm{app}}(c), \\
  \mathrm{Heat}(c) &= \eta \cdot H_{\mathrm{hist}}(c) + (1-\eta) H_{\mathrm{slot}}(c,t),
\end{align}
其中衰减 $\rho\in(0,1)$,混合系数 $\eta=0.7$。请求服从 Zipf 分布(参数 $\zeta$,可配)。

\subsection{边缘协作与预取}
邻居状态按间隔 $T_{\mathrm{sync}}$ 同步,避免覆盖半径内重复缓存,并对即将热点的内容进行预取(预留小比例缓存容量作为预取窗口)。

\subsection{容量约束下的背包选择}
设缓存容量 $C$(bits),候选内容集为 $\mathcal{C}$,价值 $v_c$、大小 $s_c$,求解:
\begin{equation}
  \max_{\mathcal{S}\subseteq \mathcal{C}} \sum_{c\in \mathcal{S}} v_c \quad \text{s.t. } \sum_{c\in \mathcal{S}} s_c \le C,
\end{equation}
其中 $v_c$ 综合命中价值、最近性、访问频率、大小惩罚与优先级权重。

\subsection{决策逻辑(与实现一致的四类动作)}
对请求内容 $c$ 与可用容量 $C_{\mathrm{free}}$:
\begin{equation}
\mathrm{action}(c) =
\begin{cases}
0, & c \in \mathrm{cache} \\
1, & \mathrm{Heat}(c)>\tau_h \land C_{\mathrm{free}}>\tau_c \\
2, & \tau_m<\mathrm{Heat}(c)\le\tau_h \quad(\text{预取})\\
3, & \text{否则在背包替换下进行置换}
\end{cases}
\end{equation}

\subsection{缓存奖励与成本整形}
定义整形奖励:
\begin{equation}
R_{\mathrm{cache}} = \alpha_{\mathrm{hit}}\, \mathrm{HitRate}
- \alpha_{\mathrm{cost}}\, \mathrm{OpCost}
- \alpha_{\mathrm{over}}\, \mathrm{OverBudget}
- \alpha_{\mathrm{energy}}\, \mathrm{Energy},
\end{equation}
并对最近性/频率/大小/优先级权重自适应调整以稳态优化。

\subsection{伪代码}
\begin{algorithm}[H]
\caption{热度+协作+背包的动态边缘缓存}
\Input{请求 $c$,节点状态,容量 $C$}
\Output{缓存动作 $\in\{0,1,2,3\}$}
update\_history($c$),update\_slot\_heat($c,t$);\\
\If{collab\_enabled \& time\_to\_sync()}{sync\_neighbors();}
$h \leftarrow \mathrm{Heat}(c)$;\\
\If{$c$ in cache}{\textbf{return} 0}
\If{$C_{\mathrm{free}} \le 0$}{\If{knapsack\_enabled}{\textbf{return} 3}\Else{\textbf{return} 0}}
\If{$h>\tau_h$ \& $C_{\mathrm{free}}>\tau_c$}{\textbf{return} 1}
\If{$\tau_m<h\le\tau_h$}{\textbf{return} 2}
\textbf{return} 0
\end{algorithm}

\subsection{与强化学习的集成}
迁移时延/成本与缓存命中收益已纳入奖励设计。若需要,可将估计的 downtime 作为显式惩罚项接入(实现中可通过迁移统计模块记录后馈入奖励)。


\subsection{关键超参数与默认值}
\begin{itemize}
  \item 迁移:$\theta_{\mathrm{over}}^{\mathrm{RSU}}{=}0.8$,$\theta_{\mathrm{under}}^{\mathrm{RSU}}{=}0.5$,$\theta_{\mathrm{over}}^{\mathrm{UAV}}{=}0.7$,$\mathrm{battery}_{\min}{=}0.30$,$d_{\max}{=}1000\,$m,$\mathrm{bandwidth}{=}10\,$Mbps,$\mathrm{cooldown}{=}10$ 步。
  \item 评分权重:$w_{\mathrm{load}}{=}0.6$,$w_{\mathrm{dist}}{=}0.4$。
  \item 缓存:$\mathrm{cache\_size}{=}1000\,$MB,Zipf $\zeta{=}0.8$,替换策略 LRU(默认),预取窗口 $10$,$\alpha_{\mathrm{hit}}{=}3.0$,$\alpha_{\mathrm{cost}}{=}1.0$,$\alpha_{\mathrm{over}}{=}5.0$,$\alpha_{\mathrm{energy}}{=}0.2$。
  \item 热度混合:$\rho{\in}(0,1)$,$\eta{=}0.7$;时间槽数 $T_{\mathrm{slot}}{=}24$。
\end{itemize}

\subsection{实验指标与评估方法}
\textbf{迁移指标}:迁移成功率、平均迁移时延 $\mathbb{E}[T_{\mathrm{mig}}]$、downtime 估计 $T_{\mathrm{quiesce}}$、回滚率、负载均衡改善率、SLA 违约率(deadline miss)。\\
\textbf{缓存指标}:缓存命中率、预取命中率、替换次数、带宽节省率、缓存能耗、超预算次数。\\
\textbf{系统指标}:端到端平均时延、能耗、任务完成率、资源利用率(CPU/带宽)。

\subsection{复杂度与开销分析}
评分计算与目标选择为 $O(|\mathcal{T}|)$;若采用背包选择,在容量离散化下动态规划为 $O(NC)$($N$ 为候选内容数,$C$ 为容量尺度),工程上可用贪心/近似以控时。迁移阶段的主要开销来自状态传输,按块大小 $b$ 与速率 $R$ 近似 $T \approx \sum_k \frac{b_k}{R_k}$;预取与协作同步为低频小批量操作,对实时路径不构成瓶颈。

\subsection{实现—论文对照表}
\begin{table}[htbp]
  \centering
  \small
  \begin{tabular}{p{4.2cm} p{7.2cm} p{4.2cm}}
    \hline
    \textbf{实现文件/类} & \textbf{论文对应术语/符号} & \textbf{论文小节/式号} \\
    \hline
    \texttt{common/enhanced\_task\_migration\_manager.py} & 任务迁移模型、迁移类型(RSU$\rightarrow$RSU/RSU$\rightarrow$UAV/UAV$\rightarrow$RSU/Vehicle-Follow/Preemptive) & §"任务迁移模型与低中断切换" \\
    MigrationType Enum & 迁移类型集合 & 同上 \\
    \_calculate\_target\_score() & 目标评分 $\mathrm{score}(s,t)$(负载+距离加权) & 式(30) \\
    \_calculate\_migration\_cost/delay/success\_prob() & $C_{\mathrm{mig}}$、$T_{\mathrm{mig}}$、$p_{\mathrm{succ}}$ 启发式 & 式(31-33) 附近 \\
    migration\_config.py & 阈值/权重/超时/冷却/带宽等配置 & "关键超参数与默认值" \\
    common/bandwidth\_allocator.py & 迁移流量带宽最小份额(保底) & "低中断切换思路(工程实现)" \\
    common/enhanced\_cache\_manager.py & 边缘缓存(LRU/LFU/FIFO/HYBRID、预取、统计、奖励) & §"边缘缓存模型与策略实现" \\
    common/dynamic\_cache\_strategy.py & 历史+时间槽热度、协作、背包、四类动作 & 式(35-37) 与决策逻辑 \\
    resource\_config.py / constraints\_config.py & 总带宽、最小份额、容量/稳定性约束等 & §系统约束与资源映射 \\
    DRL 奖励整形(相关模块) & $R_{\mathrm{cache}}$ 与全局目标中的成本项 & "缓存奖励与成本整形" \\
    config.py(Env/Algo) & 默认参数、训练相关超参 & "关键超参数与默认值" \\
    \hline
  \end{tabular}
  \caption{实现与论文表述对照表(核心模块到符号/小节的映射)。}
  \label{tab:impl-map}
\end{table}

\subsection{图表与符号引用说明}
本文关键公式与表格可按如下方式引用:公式(30)–(33) 对应迁移评分/成本/时延/成功率建模;表~\ref{tab:impl-map} 给出了实现到论文术语的小节映射;热度建模与决策逻辑对应式(35)–(37)。如需配图示例,可复用图~\ref{fig:example} 的版式放置系统流程图或缓存策略流程图。

\subsection{系统总体架构与并排流程图}
\begin{figure}[htbp]
  \centering
  \begin{subfigure}[t]{0.48\textwidth}
    \centering
    \begin{tikzpicture}[
      node distance=7mm,
      every node/.style={font=\scriptsize},
      comp/.style={rectangle,rounded corners,draw,align=center,minimum width=18mm,minimum height=5mm},
      >={Stealth[length=2mm]}
    ]
      \node[comp] (veh) {车辆/本地计算};
      \node[comp,right=of veh] (rsu) {RSU(计算+缓存)};
      \node[comp,right=of rsu] (uav) {UAV(计算)};
      \draw[->] (veh) -- node[above]{V2I} (rsu);
      \draw[->] (veh) |- node[below]{V2U} (uav);
      \draw[<->] (rsu) -- node[above]{回程/迁移} (uav);
      \node[comp,below=12mm of rsu] (cache) {缓存管理(热度/协作/背包)};
      \draw[->] (rsu) -- (cache);
      \node[comp,below=12mm of veh] (mig) {迁移管理(评分/约束)};
      \draw[->] (veh) -- (mig);
      \node[comp,below=12mm of uav] (sched) {带宽/队列/调度};
      \draw[->] (uav) -- (sched);
    \end{tikzpicture}
    \caption{系统总体架构(模块与数据流)。}
    \label{fig:system-arch}
  \end{subfigure}\hfill
  \begin{subfigure}[t]{0.48\textwidth}
    \centering
    \begin{tikzpicture}[
      node distance=7mm,
      every node/.style={font=\scriptsize},
      box/.style={rectangle,rounded corners,draw,align=center,minimum width=28mm,minimum height=5mm},
      decision/.style={diamond,aspect=2,draw,align=center,inner sep=1mm},
      >={Stealth[length=2mm]}
    ]
      \node[box] (req) {请求到达 $c$ / 触发迁移检测};
      \node[decision,below=of req] (branch) {缓存命中?};
      \node[box,right=14mm of branch] (hit) {返回(动作0)};
      \node[box,below=of branch] (score) {迁移评分/候选选择};
      \node[box,below=of score] (pre) {目标预热/带宽保底};
      \node[box,below=of pre] (final) {短静默/切换};
      \draw[->] (req) -- (branch);
      \draw[->] (branch.east) -- node[above]{是} (hit.west);
      \draw[->] (branch) -- node[left]{否} (score);
      \draw[->] (score) -- (pre);
      \draw[->] (pre) -- (final);
    \end{tikzpicture}
    \caption{缓存/迁移联合流程(简化)。}
    \label{fig:flow-side}
  \end{subfigure}
  \caption{系统总体架构与联合流程示意图。}
  \label{fig:system-arch-and-flows}
\end{figure}
\newpage
\section{优化问题定义}
\label{sec:optimization_problem}

基于前述的系统模型、通信模型和计算模型,本节将系统所面临的联合任务分配、资源调度与缓存管理问题形式化为一个多目标优化问题。该问题的核心目标是在满足各类服务质量 (QoS) 要求和物理资源限制的前提下,最小化系统的长期平均加权成本,主要包括任务完成时延和系统总能耗。

\subsection{目标函数}

基于前述的系统模型、通信模型和计算模型,本节将系统所面临的联合任务分配、资源调度与缓存管理问题形式化为一个多目标优化问题。该问题的核心目标是在满足各类服务质量 (QoS) 要求和物理资源限制的前提下,最小化系统的长期平均加权成本,主要包括**任务完成时延**、**系统总能耗**以及**任务数据丢失量**。

\subsection{目标函数}

我们的优化目标是最小化在时间跨度 $T$ 内(包含多个时隙)系统总成本的长期期望值。该成本函数是一个加权和,包含了**平均任务时延**、**总能耗**以及**任务失败率**。

设 $\mathcal{J}_{succ}^t$ 为在时隙 $t$ 成功完成的任务集合,$\mathcal{J}_{arrived}^t$ 为在时隙 $t$ 到达的任务集合,$\mathcal{J}_{failed}^t$ 为在时隙 $t$ 失败(丢弃或超过截止时间)的任务集合。优化问题 (P1) 重构为:

\begin{equation}
\label{eq:objective_function_data_loss}
\min_{\mathbf{X}, \mathbf{R}, \mathbf{Z}} \lim_{T \to \infty} \frac{1}{T} \mathbb{E} \Bigg[ \sum_{t=0}^{T-1} \Big( 
\, \omega_T \, L_{\mathrm{norm}}^t 
\; + \; \omega_E \, \frac{E_{total}^t}{E_{\mathrm{ref}}} 
\; + \; \omega_D \, \mathrm{FailRate}^t \Big) \Bigg]
\end{equation}
其中:
\begin{itemize}
    \item $\mathbf{X} = \{x_{j,n}^t\}$,$\mathbf{R} = \{b_{link}^t,\, \mu_n^t,\, \nu_{n,n'}^t\}$,$\mathbf{Z} = \{z_{j,r}^t\}$ 分别代表所有时隙内的任务分配/路由、资源管理(带宽分配、节点激活、迁移链路开关)与缓存决策变量集合。\\ 数据流量由 $\mathbf{X}$ 所诱导(例如 $x_{j,n}^t=1$ 则上传 $D_j$),无需单独定义 $\mathbf{D}$ 变量。
    \item $\omega_T, \omega_E, \omega_D$ 是非负权重系数,用于平衡时延、能耗和数据丢失量三者之间的重要性。这些权重可以根据不同的应用场景进行调整(例如,在要求高可靠性的场景中增大 $\omega_D$)。
    \item $\sum_{j \in \mathcal{J}_{succ}^t} T_{total,j}$ 是在时隙 $t$ 内所有成功完成任务的总时延。$T_{total,j}$ 是任务 $j$ 的端到端完成时延。
    \item $E_{total}^t$ 是系统在时隙 $t$ 的总能耗,由式 \eqref{eq:E_total_t_final_revised_sec7} 给出。
    \item $\sum_{j \in \mathcal{J}_{dropped}^t} D_j$ 是在时隙 $t$ 内所有被丢弃任务的**输入数据总量** (bits)。这直接量化了因系统无法及时处理而造成的信息损失。$\mathcal{J}_{dropped}^t$ 包含了因超过最大延迟 $T_{max,j}$ 或因队列容量不足而被丢弃的任务。
\end{itemize}

该目标函数旨在寻找一个长期策略,该策略能够在动态变化的环境中智能地做出决策,从而在任务完成时延、系统能耗和数据丢失率之间实现最佳权衡。
\paragraph{关于迁移决策变量的说明与优化维度补充}
为避免歧义,现明确本问题的联合优化维度(动作空间)包含下列变量;其中迁移相关变量已在(P1)中纳入,通过时延与能耗项影响目标函数,无需将"迁移成功率/缓存命中率"单独作为优化目标:
\begin{itemize}
    \item 任务路由/分配决策:$\{x_{j,n}^t\}$(车辆本地/RSU/UAV)。
    \item 迁移链路开关:$\{\nu_{n,n'}^t\}$,控制是否激活 $n\!\to\!n'$ 的迁移路径(见式\eqref{eq:Rate_mig_revised_sec5}–\eqref{eq:D_mig_revised_sec5})。
    \item 迁移数据流量:$\{D_{mig,r,r',l,p}^t\}$,控制RSU间状态/任务数据的迁移量,与带宽与流守恒共同约束(见\eqref{eq:flow_conservation})。
    \item 带宽分配:$\{b_{link}^t\}$,覆盖V2I/V2U/U2V/回程(迁移)链路(见\eqref{eq:Rate_revised_sec5}, \eqref{eq:Rate_mig_revised_sec5})。
    \item 节点激活:$\{\mu_n^t\}$,用于节能(RSU/UAV按需开关)。
    \item 缓存决策:$\{z_{j,r}^t\}$,结果级缓存的二元选择及其容量约束(见\eqref{eq:cache_capacity})。
\end{itemize}
上述变量经由约束\eqref{eq:bandwidth_up}–\eqref{eq:bandwidth_down}、\eqref{eq:cache_capacity}、\eqref{eq:queue_capacity_hard}、\eqref{eq:energy_budget}、\eqref{eq:flow_conservation}、\eqref{eq:processing_capacity} 与 \eqref{eq:sinr_soft} 等共同作用于 $T_{total}$ 与 $E_{total}$,从而隐式地对"迁移是否执行、迁往何处、分配多少带宽/数据量"进行优化。迁移造成的短暂 downtime 亦已通过迁移延迟与排队耦合进入 $T_{total}$(参见第6节与式\eqref{eq:E_total_t_final_revised_sec7})。

\paragraph{训练奖励与目标的一致性}
为与强化学习训练保持严格一致,建议将每时隙奖励定义为目标函数的相反数(与实现保持一致):
\begin{equation}
\label{eq:rl_reward_alignment}
R_t \;=\; -\Big(\, \omega_T \, L_{\mathrm{norm}}^t \; + \; \omega_E \, \tfrac{E_{total}^t}{E_{\mathrm{ref}}} \; + \; \omega_D \, \mathrm{FailRate}^t \Big).
\end{equation}
若引入辅助整形(如对负载均衡或参数探索的轻量奖励),应保证不改变总体优化方向,即不削弱对"时延、能耗、数据丢失"三者的主导权重;整形项宜小于主成本幅度的5–10%。

\paragraph{3GPP一致性与参数建议}
为确保与3GPP NR蜂窝链路建模一致(参考3GPP TR 38.901/38.306),本模型的速率/路径损耗/SINR已在第5.2节按香农容量与LoS/NLoS混合路径损耗给出。为便于复现与部署,给出可行的默认参数建议(实际值可按场景调参):
\begin{itemize}
    \item 载波频率 $f_c\in[3.3,3.8]$ GHz(典型3.5 GHz);编码效率 $\eta_{coding}\in[0.85,0.95]$;噪声谱密度 $N_0\approx -174$ dBm/Hz;
    \item 最小SINR门限 $\gamma_{\min}\in[-5,0]$ dB(URLLC场景取高,eMBB取低);
    \item 回程/迁移链路带宽 $B_{k,k'}^t$ 取决于部署(10–100 Mbps 常见),但必须计入总带宽预算与干扰模型;
    \item LoS/NLoS阴影衰落标准差按城区典型值3–8 dB(与式\eqref{eq:path_loss_LoS}–\eqref{eq:path_loss_NLoS}一致)。
\end{itemize}
上述设置保证"带宽/SINR/速率/能耗"链路符合3GPP口径,使优化在现实通信约束下成立。

\paragraph{权重选择与附加工程约束建议}
考虑到本工作的最终目标是最小化"时延、能耗、数据丢失",而缓存率/迁移成功率仅为手段,建议权重范围:$\omega_T\in[0.5,0.7]$,$\omega_E\in[0.2,0.4]$,$\omega_D\in[0.1,0.3]$(需归一化,按业务侧重点微调)。为提升系统稳定性,可在实现层面加入下列工程性约束/策略(作为软约束或调度规则):
\begin{itemize}
    \item 迁移冷却期与并发上限:同一节点相邻两次迁移间隔 $\ge T_{cool}$;系统并发迁移数 $\le M_{max}$,避免迁移风暴;
    \item 带宽最小保底:为迁移与关键V2I链路保留 $b_{min}$,避免被抢占导致SLA违约;
    \item 队列稳定性安全裕度:将\eqref{eq:queue_stability} 的阈值从1收紧到 $\rho_{max}<1$(如0.9),降低爆盘风险。
\end{itemize}
\subsection{决策变量}

为了最小化目标函数 \eqref{eq:objective_function_data_loss},系统需要在每个决策周期内确定一组最优的动作。这些动作对应于以下核心决策变量:

\begin{itemize}
    \item \textbf{任务路由与分配决策 ($\mathbf{X}^t$)}: 这是最核心的决策变量集合,决定了每个任务数据的流向和处理位置。
    \begin{itemize}
        \item $x_{j,n}^t \in \{0, 1\}$: 决定任务 $j$ 是否在时隙 $t$ 被分配给节点 $n$(包括车辆本地、RSU或UAV)进行处理。
        \item $\nu_{n,n'}^t \in \{0, 1\}$: 决定是否激活从节点 $n$ 到 $n'$ 的迁移路径,用于处理已在系统中排队的任务。
    \end{itemize}

    \item \textbf{资源管理决策 ($\mathbf{R}^t$)}: 决定如何分配有限的物理资源来支持任务处理和数据传输。
    \begin{itemize}
        \item $b_{link}^t \ge 0$: 为每个活跃的通信链路(如V2I、V2U、RSU间)分配具体的带宽资源。
        \item $\mu_n^t \in \{0, 1\}$: 决定是否激活某个计算节点 $n$ 的处理单元,用于节能控制。
    \end{itemize}

    \item \textbf{缓存管理决策 ($\mathbf{Z}^t$)}: 这是一个长期的策略性决策,旨在通过缓存常用计算结果来降低未来任务的时延和计算开销。
    \begin{itemize}
        \item $z_{j,r}^t \in \{0, 1\}$: 决定RSU $r$ 是否应该缓存任务 $j$ 的计算结果。
    \end{itemize}
\end{itemize}

这些决策变量共同构成了一个复杂的动作空间。数据流(例如卸载的数据量)是这些决策变量作用下的直接结果,而非独立的决策变量。

\textbf{约束条件}:
\begin{align}
    \text{(C1) 任务唯一分配}: \quad & \sum_{n \in \mathcal{N}} x_{j,n}^t \leq 1, && \forall j \in \mathcal{J}_t^{active} \label{eq:task_unique} \\
    %
    \text{(C2) 队列稳定性}: \quad & \sum_{i=1}^P \rho_{i,n}^t < \rho_{max}, && \forall n \in \mathcal{R} \cup \mathcal{U} \label{eq:queue_stability} \\
    %
    \text{(C3) 带宽容量}: \quad & \sum_{m \in \mathcal{N}, m \neq n} b_{n,m}^t \leq B_{n,\text{up}}^{max}, && \forall n \in \mathcal{N} \label{eq:bandwidth_up} \\
    & \sum_{m \in \mathcal{N}, m \neq n} b_{m,n}^t \leq B_{n,\text{down}}^{max}, && \forall n \in \mathcal{N} \label{eq:bandwidth_down} \\
    %
    \text{(C4) 缓存容量}: \quad & \sum_{j \in \mathcal{J}} z_{j,r}^t S_j \leq S_{cache,r}, && \forall r \in \mathcal{R} \label{eq:cache_capacity} \\
    %
    \text{(C5) 队列容量}: \quad & \sum_{l=1}^{L} \sum_{p=1}^{P} q_{n,l,p}^t \leq Q_{max,n}, && \forall n \in \mathcal{N} \label{eq:queue_capacity_hard} \\
    %
    \text{(C6) 能耗预算}: \quad & \sum_{t=0}^{T_{horizon}-1} E_{n,t}^{total} \leq E_{max,n}, && \forall n \in \mathcal{U} \label{eq:energy_budget} \\
    %
    \text{(C7) 数据流守恒}: \quad & D_{in,n}^t = D_{out,n}^t + D_{proc,n}^t + D_{loss,n}^t, && \forall n \in \mathcal{N} \label{eq:flow_conservation} \\
    %
    \text{(C8) 处理能力}: \quad & \sum_{l,p} D_{proc,n,l,p}^t \leq D_{max,n}^t, && \forall n \in \mathcal{N} \label{eq:processing_capacity} \\
    %
    \text{(C9) 通信质量}: \quad & \text{SINR}_{link}(t) \geq \gamma_{\min}, && \forall \text{active links} \\
    %
    \text{(C9) }: \quad & x_{j,n}^t \in \{0, 1\}, z_{j,r}^t \in \{0, 1\}, \mu_n^t \in \{0,1\}, \nu_{n,n'}^t \in \{0,1\} \label{eq:binary_vars} \\
    %
    \text{(C10) }: \quad & D_{\cdot}^t \geq 0, b_{link}^t \geq 0, E_{\cdot}^t \geq 0 \label{eq:non_negative_vars}
\end{align}

\textbf{软约束条件} (通过惩罚函数处理):
\begin{align}
    \text{(S1) 延迟QoS}: \quad & T_{total,j} \leq T_{max,j} \cdot \Delta t + \epsilon_{delay,j}, && \forall j \in \mathcal{J} \label{eq:delay_soft} \\
    %
    \text{(S2) 通信质量}: \quad & \text{SINR}_{link}(t) \geq \gamma_{\min} - \epsilon_{sinr,link}, && \forall \text{active links} \label{eq:sinr_soft} \\
    %
    \text{(S3) 负载均衡}: \quad & \left| \rho_n^t - \bar{\rho}^t \right| \leq \epsilon_{balance}, && \forall n \in \mathcal{R} \cup \mathcal{U} \label{eq:load_balance}
\end{align}

其中 $\epsilon_{\cdot} \geq 0$ 为松弛变量,在目标函数中添加相应惩罚项:
\begin{equation}
    \text{Penalty} = M_1 \sum_j \epsilon_{delay,j} + M_2 \sum_{link} \epsilon_{sinr,link} + M_3 \sum_n \epsilon_{balance,n}
\end{equation}
$M_1, M_2, M_3$ 为惩罚权重,通常设置为较大正数。

% 约束条件中文解释
\begin{itemize}
    \item \textbf{(C1) 任务唯一分配}: 在时隙 $t$,每个活跃任务 $\mathcal{J}_t^{active}$ 最多被分配给一个计算节点处理。
    \item \textbf{(C2) 队列稳定性}: RSU和UAV等边缘服务器的队列负载(流量强度 $\rho$)必须小于某个最大值(通常为1),以保证系统稳定运行。
    \item \textbf{(C3) 带宽容量}: 每个节点在所有上行或下行链路上分配的总带宽不能超过其最大可用带宽。
    \item \textbf{(C4) 缓存容量}: 每个RSU上缓存的所有任务结果的总大小不能超过其物理缓存容量。
    \item \textbf{(C5) 队列容量}: 每个节点中等待处理的任务数据总量不能超过其缓冲区大小,防止数据溢出。
    \item \textbf{(C6) 能耗预算}: 能量受限的节点(尤其是UAV)在一定时间范围内的总能耗不能超过其电池容量或能量预算。
    \item \textbf{(C7) 数据流守恒}: 对于每个节点,流入的数据量应等于被处理、转发(迁移)、存储或丢弃的数据量之和。
    \item \textbf{(C8) 处理能力}: 每个节点在一个时隙内能够处理的总数据量受其CPU计算能力的限制。
    \item \textbf{(C9) 变量域}: 决策变量中的任务分配、缓存决策、节点激活等必须为二元值(0或1)。
    \item \textbf{(C10) 非负性}: 决策变量中的数据量、带宽分配、能耗等物理量必须为非负数。
\end{itemize}

%% MODIFIED: E_total^t's E_fly^t component uses the simplified hovering energy
系统在时隙 $t$ 的总能耗为:
\begin{align}
    E_{total}^t = & \sum_{n \in \mathcal{V}} (E^{comp}_{n,t} + E^{tx}_{n,t}) \quad (\text{车辆能耗: 计算 + 发射}) \nonumber \\
                  & + \sum_{k \in \mathcal{R}} (E^{comp}_{k,t}  + E^{tx,mig}_{k,t} + E^{rx,mig}_{k,t}) \quad (\text{RSU能耗: 计算 (若发生) + 迁移发射/接收}) \nonumber\\ 
                  & + \sum_{u \in \mathcal{U}} (E^{comp}_{u,t} + E^{comm,t}_u + E^{fly,t}_u) \quad (\text{UAV能耗: 计算 + 通信 + 悬停})
    \label{eq:E_total_t_final_revised_sec7} 
\end{align}
其中 $E^{rx,mig}_{k,t}$ 是RSU $k$ 在时隙 $t$ 接收迁移数据的能耗。$E^{comp}_{k,t}$ 仅在RSU $k$ 实际执行计算(即任务未缓存命中)时产生。UAV的 $E^{fly,t}_u$ 为其固定的悬停能耗。

\end{document}