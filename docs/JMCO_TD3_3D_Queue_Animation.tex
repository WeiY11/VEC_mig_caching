\documentclass[aspectratio=169]{beamer}
\usepackage{tikz}
\usepackage{tikz-3dplot}
\usepackage{amsmath}
\usepackage{pgfplots}
\pgfplotsset{compat=1.17}
\usetikzlibrary{arrows.meta, positioning, shapes, calc, 3d}

\usetheme{Madrid}
\usecolortheme{default}

\title{JMCO-TD3队列系统3D可视化}
\subtitle{多优先级生命周期队列状态演化动画}
\author{VEC边缘计算系统}
\date{\today}

\begin{document}

\frame{\titlepage}

% ==================== 第1帧:队列系统概览 ====================
\begin{frame}{队列系统概览}
\begin{center}
\begin{tikzpicture}[scale=0.8]
    % 标题
    \node[font=\Large\bfseries, blue] at (6, 7) {多优先级生命周期队列矩阵 $(L \times P)$};
    
    % 绘制队列矩阵
    \foreach \l in {1,...,5} {
        \foreach \p in {1,...,4} {
            \pgfmathsetmacro{\x}{(\p-1)*2.8}
            \pgfmathsetmacro{\y}{6.5-(\l-1)*1.1}
            
            % 队列方框
            \draw[thick, rounded corners=3pt, fill=blue!\p5!white, drop shadow] 
                (\x, \y) rectangle ++(2.3, 0.9);
            
            % 队列标签
            \node[font=\large] at (\x+1.15, \y+0.45) {$Q_{\l,\p}$};
        }
    }
    
    % 轴标签
    \node[rotate=90, anchor=south, font=\Large\bfseries] at (-1.2, 4) {\textcolor{red}{生命周期 (Lifetime)}};
    \foreach \l in {1,...,5} {
        \pgfmathsetmacro{\y}{6.5-(\l-1)*1.1}
        \node[anchor=east, font=\large] at (-0.3, \y+0.45) {$l=\l$};
    }
    
    \node[font=\Large\bfseries] at (5.5, 0.8) {\textcolor{blue}{优先级 (Priority)}};
    \foreach \p in {1,...,4} {
        \pgfmathsetmacro{\x}{(\p-1)*2.8}
        \node[anchor=north, font=\large] at (\x+1.15, 1.2) {$p=\p$};
    }
    
    % 图例
    \node[draw, thick, rounded corners, fill=yellow!20, align=left, anchor=west, font=\small] at (12, 5.5) {
        \textbf{参数说明:}\\
        $L = 5$ 时隙\\
        $P = 4$ 级优先级\\
        总队列数: $20$\\[0.2cm]
        \textbf{调度策略:}\\
        1. 优先级优先\\
        2. 同级FIFO\\
        3. 非抢占式
    };
\end{tikzpicture}
\end{center}
\end{frame}

% ==================== 第2帧:3D队列视图 - 初始状态 ====================
\begin{frame}{3D队列状态视图 - 时隙 $t=0$}
\begin{center}
\tdplotsetmaincoords{70}{110}
\begin{tikzpicture}[scale=1.2, tdplot_main_coords]
    % 坐标轴
    \draw[thick, ->] (0,0,0) -- (6,0,0) node[anchor=north east]{优先级 $p$};
    \draw[thick, ->] (0,0,0) -- (0,6,0) node[anchor=north west]{生命周期 $l$};
    \draw[thick, ->] (0,0,0) -- (0,0,6) node[anchor=south]{队列长度};
    
    % 绘制3D柱状图(任务数量)
    \foreach \p in {1,...,4} {
        \foreach \l in {1,...,5} {
            \pgfmathsetmacro{\height}{1 + rand*2}
            \pgfmathsetmacro{\colorintensity}{20 + \p*15}
            
            % 3D方块
            \fill[blue!\colorintensity!white, opacity=0.8] 
                (\p, \l, 0) -- (\p+0.8, \l, 0) -- (\p+0.8, \l+0.8, 0) -- (\p, \l+0.8, 0) -- cycle;
            \fill[blue!\colorintensity!white, opacity=0.6] 
                (\p, \l, 0) -- (\p+0.8, \l, 0) -- (\p+0.8, \l, \height) -- (\p, \l, \height) -- cycle;
            \fill[blue!\colorintensity!white, opacity=0.9] 
                (\p+0.8, \l, 0) -- (\p+0.8, \l+0.8, 0) -- (\p+0.8, \l+0.8, \height) -- (\p+0.8, \l, \height) -- cycle;
            \fill[red!\colorintensity!yellow, opacity=0.7] 
                (\p, \l, \height) -- (\p+0.8, \l, \height) -- (\p+0.8, \l+0.8, \height) -- (\p, \l+0.8, \height) -- cycle;
            
            % 数量标签
            \node at (\p+0.4, \l+0.4, \height+0.3) {\scriptsize \pgfmathparse{int(\height)}\pgfmathresult};
        }
    }
    
    % 刻度标签
    \foreach \p in {1,...,4} {
        \node at (\p+0.4, 0, -0.5) {\small $\p$};
    }
    \foreach \l in {1,...,5} {
        \node at (0, \l+0.4, -0.5) {\small $\l$};
    }
\end{tikzpicture}
\end{center}
\vspace{-0.5cm}
\begin{center}
    \textbf{初始队列状态:}系统启动,任务随机分布于各优先级队列
\end{center}
\end{frame}

% ==================== 第3帧:生命周期演化动画 - 帧1 ====================
\begin{frame}{队列演化:时隙 $t \to t+1$ (1/5)}
\begin{columns}[T]
\column{0.48\textwidth}
\begin{center}
\textbf{时隙 $t$}\\[0.3cm]
\tdplotsetmaincoords{70}{110}
\begin{tikzpicture}[scale=0.9, tdplot_main_coords]
    \draw[thick, ->] (0,0,0) -- (5,0,0);
    \draw[thick, ->] (0,0,0) -- (0,5,0);
    \draw[thick, ->] (0,0,0) -- (0,0,5);
    
    % 绘制队列状态 - 时隙t
    \foreach \p in {1,...,4} {
        \foreach \l in {1,...,5} {
            \pgfmathsetmacro{\height}{0.5 + (6-\l)*0.5 + rand*0.3}
            \pgfmathsetmacro{\colorintensity}{20 + \p*15}
            
            \fill[blue!\colorintensity!white, opacity=0.8] 
                (\p, \l, 0) -- (\p+0.7, \l, 0) -- (\p+0.7, \l+0.7, 0) -- (\p, \l+0.7, 0) -- cycle;
            \fill[blue!\colorintensity!white, opacity=0.9] 
                (\p+0.7, \l, 0) -- (\p+0.7, \l+0.7, 0) -- (\p+0.7, \l+0.7, \height) -- (\p+0.7, \l, \height) -- cycle;
            \fill[red!\colorintensity!yellow, opacity=0.7] 
                (\p, \l, \height) -- (\p+0.7, \l, \height) -- (\p+0.7, \l+0.7, \height) -- (\p, \l+0.7, \height) -- cycle;
        }
    }
\end{tikzpicture}
\end{center}

\column{0.48\textwidth}
\begin{center}
\textbf{时隙 $t+1$}\\[0.3cm]
\tdplotsetmaincoords{70}{110}
\begin{tikzpicture}[scale=0.9, tdplot_main_coords]
    \draw[thick, ->] (0,0,0) -- (5,0,0);
    \draw[thick, ->] (0,0,0) -- (0,5,0);
    \draw[thick, ->] (0,0,0) -- (0,0,5);
    
    % 绘制队列状态 - 时隙t+1(生命周期递减)
    \foreach \p in {1,...,4} {
        \foreach \l in {1,...,4} {
            \pgfmathsetmacro{\lprev}{\l+1}
            \pgfmathsetmacro{\height}{0.5 + (6-\lprev)*0.5 + rand*0.3}
            \pgfmathsetmacro{\colorintensity}{20 + \p*15}
            
            \fill[green!\colorintensity!white, opacity=0.8] 
                (\p, \l, 0) -- (\p+0.7, \l, 0) -- (\p+0.7, \l+0.7, 0) -- (\p, \l+0.7, 0) -- cycle;
            \fill[green!\colorintensity!white, opacity=0.9] 
                (\p+0.7, \l, 0) -- (\p+0.7, \l+0.7, 0) -- (\p+0.7, \l+0.7, \height) -- (\p+0.7, \l, \height) -- cycle;
            \fill[red!\colorintensity!yellow, opacity=0.7] 
                (\p, \l, \height) -- (\p+0.7, \l, \height) -- (\p+0.7, \l+0.7, \height) -- (\p, \l+0.7, \height) -- cycle;
        }
    }
    
    % l=5 新到达任务
    \foreach \p in {1,...,4} {
        \pgfmathsetmacro{\height}{0.3 + rand*0.2}
        \pgfmathsetmacro{\colorintensity}{20 + \p*15}
        \fill[yellow!\colorintensity!white, opacity=0.9] 
            (\p, 5, 0) -- (\p+0.7, 5, 0) -- (\p+0.7, 5+0.7, 0) -- (\p, 5+0.7, 0) -- cycle;
        \fill[yellow!\colorintensity!white, opacity=0.9] 
            (\p+0.7, 5, 0) -- (\p+0.7, 5+0.7, 0) -- (\p+0.7, 5+0.7, \height) -- (\p+0.7, 5, \height) -- cycle;
        \fill[yellow, opacity=0.8] 
            (\p, 5, \height) -- (\p+0.7, 5, \height) -- (\p+0.7, 5+0.7, \height) -- (\p, 5+0.7, \height) -- cycle;
    }
\end{tikzpicture}
\end{center}
\end{columns}

\vspace{0.3cm}
\begin{center}
\begin{tikzpicture}
    \draw[->, ultra thick, red] (0,0) -- (3,0);
    \node[above, font=\small] at (1.5, 0) {生命周期递减:$l \to l-1$};
    \node[below, font=\small, align=center] at (1.5, 0) {绿色:任务迁移\\黄色:新到达任务};
\end{tikzpicture}
\end{center}
\end{frame}

% ==================== 第4帧:任务丢弃机制 ====================
\begin{frame}{任务丢弃机制:生命周期耗尽}
\begin{center}
\tdplotsetmaincoords{70}{110}
\begin{tikzpicture}[scale=1.1, tdplot_main_coords]
    % 坐标轴
    \draw[thick, ->] (0,0,0) -- (5,0,0) node[anchor=north east]{$p$};
    \draw[thick, ->] (0,0,0) -- (0,6,0) node[anchor=north west]{$l$};
    \draw[thick, ->] (0,0,0) -- (0,0,5) node[anchor=south]{队列长度};
    
    % l=1 队列(即将丢弃)
    \foreach \p in {1,...,4} {
        \pgfmathsetmacro{\height}{1 + rand*1}
        \fill[red!70, opacity=0.9] 
            (\p, 1, 0) -- (\p+0.7, 1, 0) -- (\p+0.7, 1+0.7, 0) -- (\p, 1+0.7, 0) -- cycle;
        \fill[red!70, opacity=0.9] 
            (\p+0.7, 1, 0) -- (\p+0.7, 1+0.7, 0) -- (\p+0.7, 1+0.7, \height) -- (\p+0.7, 1, \height) -- cycle;
        \fill[red!90, opacity=0.9] 
            (\p, 1, \height) -- (\p+0.7, 1, \height) -- (\p+0.7, 1+0.7, \height) -- (\p, 1+0.7, \height) -- cycle;
        
        % 警告符号
        \node[font=\Large, red] at (\p+0.35, 1+0.35, \height+0.5) {$\times$};
    }
    
    % 其他队列(正常)
    \foreach \p in {1,...,4} {
        \foreach \l in {2,...,5} {
            \pgfmathsetmacro{\height}{0.8 + rand*0.8}
            \pgfmathsetmacro{\colorintensity}{30 + \p*10}
            \fill[blue!\colorintensity!white, opacity=0.7] 
                (\p, \l, 0) -- (\p+0.7, \l, 0) -- (\p+0.7, \l+0.7, 0) -- (\p, \l+0.7, 0) -- cycle;
            \fill[blue!\colorintensity!white, opacity=0.8] 
                (\p+0.7, \l, 0) -- (\p+0.7, \l+0.7, 0) -- (\p+0.7, \l+0.7, \height) -- (\p+0.7, \l, \height) -- cycle;
            \fill[blue!\colorintensity!cyan, opacity=0.7] 
                (\p, \l, \height) -- (\p+0.7, \l, \height) -- (\p+0.7, \l+0.7, \height) -- (\p, \l+0.7, \height) -- cycle;
        }
    }
    
    % 丢弃箭头
    \draw[->, ultra thick, red, line width=2pt] (2.5, 1, -1) -- (2.5, 1, -2.5);
    \node[anchor=north, font=\large\bfseries, red] at (2.5, 1, -2.8) {丢弃 (Dropped)};
\end{tikzpicture}
\end{center}

\vspace{-0.3cm}
\begin{block}{丢弃机制}
\begin{itemize}
    \item \textcolor{red}{红色方块}:$l=1$ 队列,任务生命周期即将耗尽
    \item 时隙结束时:$l=1 \to l=0$,所有任务被标记为丢弃
    \item 丢弃率公式:$P_{drop} = N_{drop} / N_{total}$
\end{itemize}
\end{block}
\end{frame}

% ==================== 第5帧:M/M/1队列模型 ====================
\begin{frame}{M/M/1非抢占式优先级队列模型}
\begin{center}
\begin{tikzpicture}[scale=0.9]
    % 标题
    \node[font=\Large\bfseries, blue] at (6, 6.5) {M/M/1非抢占式优先级队列};
    
    % 到达过程
    \foreach \p/\y in {1/5, 2/4, 3/3, 4/2} {
        \node[anchor=east, font=\large] at (0, \y) {优先级 $\p$:};
        \draw[->, ultra thick, blue] (0.5, \y) -- node[above, font=\small] {$\lambda_{\p}$ (泊松到达)} (3, \y);
    }
    
    % 队列缓冲区
    \draw[thick, rounded corners=5pt, fill=yellow!20, drop shadow] (3.5, 0.5) rectangle (8, 6);
    \node[anchor=north, font=\Large\bfseries] at (5.75, 0.3) {队列缓冲区};
    
    % 队列中的任务
    \foreach \p/\y in {1/5.2, 2/4.2, 3/3.2, 4/2.2} {
        \node[anchor=west, font=\small] at (3.7, \y) {$p=\p$:};
        \foreach \i in {1,...,4} {
            \pgfmathsetmacro{\x}{4.3+\i*0.6}
            \draw[thick, fill=blue!\p5!white, rounded corners=2pt] 
                (\x, \y-0.2) rectangle ++(0.5, 0.4);
        }
    }
    
    % 服务器
    \draw[thick, rounded corners=5pt, fill=green!30, drop shadow] (9, 2.5) rectangle (11.5, 4.5);
    \node[align=center, font=\large\bfseries] at (10.25, 3.5) {服务器\\$\mu$};
    
    % 当前服务任务
    \draw[thick, fill=red!60, rounded corners=3pt] (9.3, 2.8) rectangle (11.2, 4.2);
    \node[font=\normalsize, white] at (10.25, 3.5) {Task};
    
    % 调度策略
    \draw[->, ultra thick, red, line width=2pt] (8.2, 3.5) -- (8.8, 3.5);
    \node[anchor=south, font=\small, align=center, red] at (8.5, 3.6) {非抢占式\\调度};
    
    % 离开过程
    \draw[->, ultra thick, green!70!black, line width=2pt] (11.7, 3.5) -- node[above, font=\small] {完成 (指数服务)} (14, 3.5);
    
    % 等待时间公式
    \node[draw, thick, rounded corners, fill=blue!10, align=left, anchor=north west, font=\small] at (3.5, -0.5) {
        \textbf{等待时间预测公式:}\\[0.1cm]
        $\displaystyle W_p = \frac{1}{\mu} \cdot \frac{\sum_{i=1}^{p} \rho_i}{(1 - \sum_{i=1}^{p-1} \rho_i)(1 - \sum_{i=1}^{p} \rho_i)}$\\[0.1cm]
        其中 $\rho_i = \lambda_i / \mu$ (流量强度)
    };
    
    % 稳定性条件
    \node[draw, thick, rounded corners, fill=red!10, align=left, anchor=north west, font=\small] at (3.5, -2.5) {
        \textbf{稳定性条件:} $\rho = \sum_{i=1}^{P} \rho_i < 1$
    };
\end{tikzpicture}
\end{center}
\end{frame}

% ==================== 第6帧:等待时间对比 ====================
\begin{frame}{各优先级等待时间对比}
\begin{center}
\begin{tikzpicture}
    \begin{axis}[
        width=13cm,
        height=7cm,
        xlabel={\Large 优先级 (Priority)},
        ylabel={\Large 平均等待时间 (ms)},
        ymode=log,
        xmin=0.5, xmax=4.5,
        ymin=10, ymax=1000,
        xtick={1,2,3,4},
        xticklabels={EMERGENCY,CRITICAL,HIGH,NORMAL},
        x tick label style={font=\large},
        y tick label style={font=\large},
        ymajorgrids=true,
        grid style=dashed,
        legend pos=north west,
        legend style={font=\large},
    ]
    
    % 理论值
    \addplot[color=blue, mark=*, mark size=4pt, line width=2pt] 
        coordinates {(1,45) (2,61) (3,111) (4,500)};
    \addlegendentry{理论预测 (M/M/1)}
    
    % 模拟值
    \addplot[color=red, mark=square*, mark size=4pt, line width=2pt, dashed] 
        coordinates {(1,48) (2,65) (3,118) (4,520)};
    \addlegendentry{仿真结果}
    
    % 误差带
    \addplot[color=blue, opacity=0.2, forget plot, fill=blue!20] 
        coordinates {(1,43) (2,58) (3,105) (4,480)} 
        \closedcycle;
    
    \end{axis}
\end{tikzpicture}
\end{center}

\vspace{-0.3cm}
\begin{block}{观察结论}
\begin{itemize}
    \item 等待时间随优先级降低呈\textbf{指数增长}
    \item 理论预测与仿真结果高度吻合(误差 $< 5\%$)
    \item 低优先级任务面临"饥饿"风险($p=4$时达500ms)
\end{itemize}
\end{block}
\end{frame}

% ==================== 第7帧:系统性能指标 ====================
\begin{frame}{队列系统性能指标}
\begin{center}
\begin{tikzpicture}[scale=0.9]
    % 性能指标仪表盘
    \node[font=\Large\bfseries, blue] at (6, 7) {系统性能监控仪表盘};
    
    % 利用率
    \draw[thick, fill=green!20, rounded corners] (0, 4) rectangle (4, 6.5);
    \node[anchor=north, font=\large\bfseries] at (2, 6.3) {系统利用率};
    \draw[ultra thick, green!70!black] (0.5, 5.5) arc (180:288:1.5);
    \node[font=\huge] at (2, 4.8) {83.3\%};
    \node[font=\small] at (2, 4.3) {$\rho = 0.833$};
    
    % 平均队列长度
    \draw[thick, fill=blue!20, rounded corners] (4.5, 4) rectangle (8.5, 6.5);
    \node[anchor=north, font=\large\bfseries] at (6.5, 6.3) {平均队列长度};
    \node[font=\huge] at (6.5, 5) {5.0};
    \node[font=\small] at (6.5, 4.3) {任务 (Little定律)};
    
    % 丢弃率
    \draw[thick, fill=red!20, rounded corners] (9, 4) rectangle (13, 6.5);
    \node[anchor=north, font=\large\bfseries] at (11, 6.3) {任务丢弃率};
    \node[font=\huge, red] at (11, 5) {3.2\%};
    \node[font=\small] at (11, 4.3) {生命周期耗尽};
    
    % Little定律说明
    \node[draw, thick, rounded corners, fill=yellow!20, align=left, font=\small] at (6.5, 2.5) {
        \textbf{Little定律:} $L = \lambda \cdot W$\\[0.1cm]
        $L_1 = 2 \times 0.045 = 0.09$ (EMERGENCY)\\
        $L_2 = 4 \times 0.061 = 0.24$ (CRITICAL)\\
        $L_3 = 6 \times 0.111 = 0.67$ (HIGH)\\
        $L_4 = 8 \times 0.500 = 4.00$ (NORMAL)\\[0.1cm]
        总计: $L = 5.00$ 个任务
    };
\end{tikzpicture}
\end{center}
\end{frame}

% ==================== 第8帧:调度策略演示 ====================
\begin{frame}{非抢占式优先级调度策略}
\begin{center}
\begin{tikzpicture}[scale=0.85]
    % 时间轴
    \draw[thick, ->] (0, 0) -- (14, 0) node[anchor=west] {时间};
    \foreach \t in {0,2,...,12} {
        \draw (\t, -0.1) -- (\t, 0.1);
        \node[anchor=north] at (\t, -0.1) {\small $t_{\t}$};
    }
    
    % 任务到达和服务过程
    % Task 1 (p=3, 到达t0)
    \draw[thick, fill=blue!30] (0, 1) rectangle (3, 1.8);
    \node at (1.5, 1.4) {\small Task1 (p=3)};
    
    % Task 2 (p=1, 到达t1, 插队)
    \draw[->, thick, red] (2, 3) -- (2, 2);
    \node[anchor=south, red] at (2, 3) {\small 到达 (p=1)};
    \draw[thick, fill=red!40] (3, 1) rectangle (5, 1.8);
    \node at (4, 1.4) {\small Task2 (p=1)};
    
    % Task 3 (p=2, 到达t3)
    \draw[->, thick, orange] (4, 3) -- (4, 2);
    \node[anchor=south, orange] at (4, 3) {\small 到达 (p=2)};
    \draw[thick, fill=orange!40] (5, 1) rectangle (7, 1.8);
    \node at (6, 1.4) {\small Task3 (p=2)};
    
    % Task 4 (p=3, 继续)
    \draw[thick, fill=blue!30] (7, 1) rectangle (9, 1.8);
    \node at (8, 1.4) {\small Task1 (继续)};
    
    % 说明文字
    \node[draw, thick, rounded corners, fill=yellow!20, align=left, anchor=west, font=\small] at (0, -2) {
        \textbf{非抢占式调度规则:}\\
        1. Task1 (p=3) 在 $t_0$ 开始处理\\
        2. Task2 (p=1) 在 $t_1$ 到达,但\textcolor{red}{不中断}当前任务\\
        3. Task1 完成后,\textcolor{red}{优先处理}Task2 (p=1)\\
        4. 然后处理Task3 (p=2),最后回到Task1剩余部分
    };
\end{tikzpicture}
\end{center}
\end{frame>

% ==================== 第9帧:JMCO-TD3架构 ====================
\begin{frame}{JMCO-TD3混合DRL+启发式架构}
\begin{center}
\begin{tikzpicture}[scale=0.75]
    % 标题
    \node[font=\Large\bfseries, blue] at (6, 8) {JMCO-TD3混合优化框架};
    
    % 任务到达
    \node[draw, thick, rounded corners, fill=blue!20, minimum width=3cm, minimum height=1cm] (task) at (6, 7) {任务到达};
    
    % 启发式任务分类
    \node[draw, thick, rounded corners, fill=orange!30, minimum width=4cm, minimum height=1cm] (classify) at (6, 5.5) {启发式任务分类\\(4级延迟分类)};
    \draw[->, ultra thick] (task) -- (classify);
    
    % DRL决策
    \node[draw, thick, rounded corners, fill=green!30, minimum width=4cm, minimum height=1.2cm] (drl) at (6, 3.8) {DRL卸载决策\\(TD3智能体)};
    \draw[->, ultra thick] (classify) -- (drl);
    
    % 启发式执行(并行)
    \node[draw, thick, rounded corners, fill=purple!30, minimum width=3cm, minimum height=1cm] (cache) at (2, 2) {启发式缓存\\(热度+Zipf)};
    \node[draw, thick, rounded corners, fill=purple!30, minimum width=3cm, minimum height=1cm] (migrate) at (10, 2) {启发式迁移\\(负载感知)};
    
    \draw[->, thick] (drl) -- (cache);
    \draw[->, thick] (drl) -- (migrate);
    
    % 队列系统
    \node[draw, thick, rounded corners, fill=yellow!30, minimum width=5cm, minimum height=1cm] (queue) at (6, 0.5) {多优先级生命周期队列};
    
    \draw[->, thick] (cache) -- (queue);
    \draw[->, thick] (migrate) -- (queue);
    
    % 反馈
    \draw[->, thick, dashed, red] (queue.east) -- ++(2,0) |- (drl.east);
    \node[anchor=west, red, font=\small] at (8.5, 1.5) {奖励反馈};
\end{tikzpicture}
\end{center}
\end{frame}

% ==================== 第10帧:总结 ====================
\begin{frame}{总结}
\begin{block}{多优先级生命周期队列模型创新点}
\begin{enumerate}
    \item \textbf{二维队列管理}:生命周期 $\times$ 优先级
    \item \textbf{M/M/1理论预测}:精确的等待时间估计
    \item \textbf{非抢占式调度}:保证服务连续性
    \item \textbf{生命周期演化}:自动任务老化与丢弃
\end{enumerate}
\end{block}

\begin{block}{与JMCO-TD3的集成}
\begin{itemize}
    \item 队列状态作为DRL观测空间的一部分
    \item 等待时间预测指导任务分配决策
    \item 负载因子触发迁移机制
    \item 优先级调整实现自适应调度
\end{itemize}
\end{block}

\vspace{0.3cm}
\begin{center}
\begin{tikzpicture}
    \node[draw, thick, rounded corners, fill=green!20, font=\large\bfseries, minimum width=10cm, minimum height=1.2cm] at (0, 0) {
        队列模型 + DRL决策 = 高效VEC系统
    };
\end{tikzpicture}
\end{center}
\end{frame}

\end{document}



